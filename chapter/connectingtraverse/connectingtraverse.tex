%!TEX root = ../../clcxsj.tex

\chapter{附合导线近似平差程序设计}

虽然GPS的广泛应用使传统的控制测量技术应用场景日益减少,但附合导线在一些工程中应用仍然非常广泛,如地铁、
隧道或建筑物密集的城区中。在导线平差数据处理中,由于近似平差法不受边角权的影响,能
最大程度的保持数据的真实性,仍然是数据检查的重要手段和数据处理的重要方法。

\section{程序功能分析}

\subsection{测绘专业背景知识与测量数据的组织}
在单导线测量中,常用的测量方法是测回法,测量仪器为全站仪,可能一个测回或多个测回。
因此测绘工程人员更加熟悉的角度数据组织格式为:

\begin{verbatim}
测站点,第一个照准点,第二个照准点,角度观测值
\end{verbatim}

许多的测绘人员将其按照水准测量的方式把第一个照准点称为后视点,第二个照准点称为前视点,
而且全站仪能直接测量平距, 近似平差时也是按照推算线路进行的,因此将测站至第二个照准点
的平距与角度观测数据组织到一起,形成如下的导线观测数据格式:

\begin{verbatim}
后视点, 测站点, 前视点, 水平角观测值, 前视水平距离
\end{verbatim}

如下形式的观测数据即为上面数据格式的一个简单例子:

\begin{verbatim}
#测角中误差mB,导线全长相对闭合差限差分母K
mB, 20
K, 4000

#Name, X, Y, H
D01, 3805820.521, 333150.649, 0
D02, 3805813.062, 333067.961, 0

#Start, Station, End, Angle, Distance
D02, D01, D04, 95.2340, 130.779
D01, D04, D03, 90.1217, 87.851
D04, D03, D02, 87.3918, 138.998
D03, D02, D01, 86.4453, -1
\end{verbatim}

上面的数据格式中以``\#''开头的行视为数据文件的注释行,数据文件由三部分组成:
导线精度信息部分,已知控制点部分,导线观测值部分。每部分由至少一个空行分隔开,
每个部分内部不能有空行,每行的数据项由英文逗号分隔开。

角度按DDD.MMSSS形式组织为double类型的数据,如果没有距离观测值则输为0或-1。

\subsection{程序基本功能}

我们设计程序的基本功能如下:
\begin{itemize}
\item 能自然处理角度与距离测量数据;
\item  能计算角度闭合差与限差;
\item  能计算导线全长相对闭合差;
\item  能显示中间计算过程与数据;
\item  数据录入具有容错与提示功能;
\item  能导出计算成果为文本文件和Excel文件;
\item  能绘制控制网略图,能将网图输出为DXF文件;
\end{itemize}

我们设计界面如图\ref{fig:ctUI01}所示:

\begin{figure}[htbp]
	\centering
	\includegraphics[scale=0.8]{chapter/connectingtraverse/ctUI01.png}
	\caption{附合导线界面功能图}
	\label{fig:ctUI01}
\end{figure}

\section{实现代码}

 \begin{lstlisting}[language=xml]
<Window x:Class="SurApp.MainWindow"
xmlns="http://schemas.microsoft.com/winfx/2006/xaml/presentation"
xmlns:x="http://schemas.microsoft.com/winfx/2006/xaml"
xmlns:local="clr-namespace:SurApp.models"
Title="附合导线近似平差" Height="350" Width="525"
WindowState="Maximized">

<DockPanel LastChildFill="True">
<Menu  x:Name="mainmenu" DockPanel.Dock="Top" Background="AliceBlue">
<MenuItem Header="文件(F)">
<MenuItem Header="新建附合导线..." Click="NewMenuItem_Click" />
<MenuItem Header="打开附合导线文件..." Click="OpenMenuItem_Click" />
<MenuItem Header="保存附合导线文件..."  Click="SaveMenuItem_Click" />
<Separator />
<MenuItem Header="导入文本数据" Click="ImportSPointMenuItem_Click"/>
<MenuItem Header="导出文本数据" Click="OutputSPointMenuItem_Click"/>
<Separator />
<MenuItem Header="导出为DXF文件" />
</MenuItem>

<MenuItem Header="数据处理(D)">
<MenuItem Header="附合导线平差" Click="AdjustMenuItem_Click" />
<MenuItem Header="平差成果" Click="AdjustResultMenuItem_Click" />
</MenuItem>

<MenuItem Header="测量(S)">
<MenuItem Header="角度弧度转换" Click="DMS2RADMenuItem_Click" />
<MenuItem Header="坐标方位角计算" Click="AzimuthMenuItem_Click" />
</MenuItem>
</Menu>
<StatusBar x:Name="statusBar" DockPanel.Dock="Bottom" Height="25"  Background="AliceBlue"/>
<Grid>
<Grid.ColumnDefinitions>
<ColumnDefinition Width="75*"/>
<ColumnDefinition Width="5"/>
<ColumnDefinition Width="170*"/>
</Grid.ColumnDefinitions>
<Border BorderThickness="2" Background="Goldenrod" Grid.Column="0">
<Grid x:Name="leftGrid" >
<Grid.RowDefinitions>
<RowDefinition Height="20"/>
<RowDefinition Height="100*"/>
<RowDefinition Height="20"/>
<RowDefinition Height="40*"/>
<RowDefinition Height="20"/>
<RowDefinition Height="100*"/>
</Grid.RowDefinitions>
<TextBlock Text="控制点数据" Grid.Row="0" TextAlignment="Center" Margin="2" />
<DataGrid x:Name="ctrPointDataGrid" Grid.Row="1" AutoGenerateColumns="False" Margin="2" ItemsSource="{Binding CtrPoints}" >
<DataGrid.Columns>
<DataGridTextColumn Header="点名" Binding="{Binding Name}" MinWidth="40"/>
<DataGridTextColumn Header="X" Binding="{Binding X , StringFormat={}{0:##0.###}}" MinWidth="60"/>
<DataGridTextColumn Header="Y" Binding="{Binding Y, StringFormat={}{0:##0.###}}" MinWidth="60" />
<DataGridTextColumn Header="Z" Binding="{Binding Z, StringFormat={}{0:##0.###}}" MinWidth="60"/>
<DataGridTextColumn Header="已知点" Binding="{Binding IsKnown}" MinWidth="30"/>
</DataGrid.Columns>
</DataGrid>

<TextBlock Text="坐标方位角数据" Grid.Row="2" TextAlignment="Center" Margin="2" />
<DataGrid x:Name="azimuthDataGrid" Grid.Row="3" AutoGenerateColumns="False" Margin="2" ItemsSource="{Binding KnownAzimuthes}" >
<DataGrid.Columns>
<DataGridTextColumn Header="起点" Binding="{Binding StartPnt.Name}" MinWidth="60"/>
<DataGridTextColumn Header="方向点" Binding="{Binding EndPnt.Name}" MinWidth="60"/>
<DataGridTextColumn Header="坐标方位角" Binding="{Binding Azimuth}" MinWidth="100" />
</DataGrid.Columns>
</DataGrid>

<TextBlock Text="观测值数据" Grid.Row="4" TextAlignment="Center" Margin="2"/>
<DataGrid x:Name="obsValueDataGrid" Grid.Row="5" AutoGenerateColumns="False"  Margin="2" ItemsSource="{Binding ObsValues}">
<DataGrid.Columns>
<DataGridTextColumn Header="测站" Binding="{Binding StnPnt.Name}" MinWidth="40"/>
<DataGridTextColumn Header="起点" Binding="{Binding StartPnt.Name}" MinWidth="40"/>
<DataGridTextColumn Header="终点" Binding="{Binding EndPnt.Name}" MinWidth="40" />
<DataGridTextColumn Header="观测角" Binding="{Binding DmsAngle}" MinWidth="60"/>
<DataGridTextColumn Header="平距" Binding="{Binding Distance}" MinWidth="60"/>
</DataGrid.Columns>
</DataGrid>

</Grid>
</Border>

<GridSplitter Background="Red" Width="5" Grid.Column="1" HorizontalAlignment="Stretch"/>

<Border BorderThickness="2" Grid.Column="2">
<local:DrawingCanvas x:Name="figureCanvas" SizeChanged="figureCanvas_SizeChanged">
<!--<Canvas.RenderTransform>
<TransformGroup>
<ScaleTransform x:Name="scaleTransform" ScaleX="1" ScaleY="-1" />
<TranslateTransform X ="0"  Y="{Binding ActualHeight, RelativeSource={RelativeSource AncestorType=Canvas}}" />
</TransformGroup>
</Canvas.RenderTransform>-->

<!--<Rectangle Height="50" Width="50" Fill="Red" Stroke="Blue" StrokeThickness="2" Canvas.Left="50" Canvas.Top="50" />

<Rectangle Height="50" Width="50" Fill="#CCCCCCFF" Stroke="Blue" StrokeThickness="2" Canvas.Left="50" Canvas.Top="50" >
<Rectangle.RenderTransform>
<TranslateTransform X="50" Y="50" />
</Rectangle.RenderTransform>
</Rectangle>-->
</local:DrawingCanvas>
</Border>
</Grid>
</DockPanel>

</Window>
 \end{lstlisting}

 \begin{lstlisting}[language=C]
using System;

namespace SurApp.models
{
[Serializable]
public class CtrPoint : ZXY.SPoint
{
private int isKnown = 0;

/// <summary>
/// 是否已知点(0:待定点, 1:已知点)
/// </summary>
public int IsKnown
	{
		get { return isKnown; }
		set
			{
				isKnown = value;
				this.RaisePropertyChange("IsKnown");
			}
	}

public CtrPoint() { }

public CtrPoint(string name, double x, double y, double z, int isKnown) : base(name, x, y, z)
{
		this.isKnown = isKnown;
	}

public override string ToString()
{
return string.Format("{0},{1},{2},{3}", Name, X, Y, Z);
}
}
}

using System;

namespace SurApp.models
{
[Serializable]
public class ObsValue : ZXY.NotificationObject
{
private CtrPoint stnPnt;
public CtrPoint StnPnt
	{
		get { return stnPnt; }
		set
			{
				stnPnt = value;
				this.RaisePropertyChange("StnPnt");
			}
	}

private CtrPoint startPnt;
public CtrPoint StartPnt
	{
		get { return startPnt; }
		set
			{
				startPnt = value;
				this.RaisePropertyChange("StartPnt");
			}
	}

private CtrPoint endPnt;
public CtrPoint EndPnt
	{
		get { return endPnt; }
		set
			{
				endPnt = value;
				this.RaisePropertyChange("EndPnt");
			}
	}

/// <summary>
///  观测角度值(单位:弧度)
/// </summary>
private double angleValue;

/// <summary>
/// 观测角度值(单位:弧度)
/// </summary>
public double AngleValue
	{
		get { return angleValue; }
		set
			{
				angleValue = value;
				this.RaisePropertyChange("DmsAngle");
				this.RaisePropertyChange("AngleValue");
			}
	}

/// <summary>
/// 观测角度值(单位:度分秒)
/// </summary>
public double DmsAngle
	{
		get { return ZXY.SMath.RAD2DMS(angleValue); }
		set
			{
				angleValue = ZXY.SMath.DMS2RAD(value);
				this.RaisePropertyChange("DmsAngle");
				this.RaisePropertyChange("AngleValue");
			}
	}


private double distance;
public double Distance
	{
		get { return distance; }
		set
			{
				distance = value;
				this.RaisePropertyChange("Distance");
			}
	}

public ObsValue() { }

public ObsValue(CtrPoint stnPnt, CtrPoint startPnt, CtrPoint endPnt,
double angleValue, double distance)
{
		this.stnPnt = stnPnt;
		this.startPnt = startPnt;
		this.endPnt = endPnt;
		this.DmsAngle = angleValue;
		this.distance = distance;
	}

private double vB; //角度改正数
public double VB
	{
		get { return vB; }
		set
			{
				vB = value;
				this.RaisePropertyChange("VB");
			}
	}

private double angleV; //改正后角度
public double AngleV
	{
		get { return angleV; }
		set
			{
				angleV = value;
				this.RaisePropertyChange("AngleV");
			}
	}

private double azimuth;
public double Azimuth
	{
		get { return azimuth; }
		set
			{
				azimuth = value;
				this.RaisePropertyChange("Azimuth");
			}
	}

private double dx;
public double DX
	{
		get { return dx; }
		set
			{
				dx = value;
				this.RaisePropertyChange("DX");
			}
	}

private double dy;
public double DY
	{
		get { return dy; }
		set
			{
				dy = value;
				this.RaisePropertyChange("DY");
			}
	}

private double vdx;
public double VDX
	{
		get { return vdx; }
		set
			{
				vdx = value;
				this.RaisePropertyChange("VDX");
			}
	}

private double vdy;
public double VDY
	{
		get { return vdy; }
		set
			{
				vdy = value;
				this.RaisePropertyChange("VDY");
			}
	}

public override string ToString()
{
return string.Format("{0},{1},{2},{3},{4}",
startPnt.Name, startPnt.Name, endPnt.Name,
DmsAngle, distance);
}
}
}

using System;
using System.Collections.Generic;
using System.Collections.ObjectModel;
using System.Windows.Media;
using ZXY;

namespace SurApp.models
{
/// <summary>
/// 用于已知边的坐标方位角信息
/// </summary>
[Serializable]
public class KnownAzimuth : NotificationObject
{
/// <summary>
/// 坐标方位角,单位:弧度
/// </summary>
public double azimuth;
public double Azimuth
	{
		get { return azimuth; }
		set
			{
				azimuth = value;
				this.RaisePropertyChange("Azimuth");
			}
	}

public CtrPoint startPnt; //坐标方位角的起点
public CtrPoint StartPnt
	{
		get { return startPnt; }
		set
			{
				startPnt = value;
				this.RaisePropertyChange("StartPnt");
			}
	}

public CtrPoint endPnt; //坐标方位角的方向点
public CtrPoint EndPnt
	{
		get { return endPnt; }
		set
			{
				endPnt = value;
				this.RaisePropertyChange("EndPnt");
			}
	}

public KnownAzimuth()
{
		startPnt = null;
		endPnt = null;
		azimuth = 0;
	}

public KnownAzimuth(CtrPoint startPnt, CtrPoint endPnt, double az)
{
		this.startPnt = startPnt;
		this.endPnt = endPnt;
		this.azimuth = az;
	}

public override string ToString()
{
if (startPnt == null || endPnt == null) return "~~~";
else
return string.Format("{0},{1},{2}", startPnt.Name, endPnt.Name, ZXY.SMath.RAD2DMS(azimuth));
}
}

[Serializable]
public class CtrNet : NotificationObject
{
private ObservableCollection<KnownAzimuth> knownAzimuthes = new ObservableCollection<KnownAzimuth>();
public ObservableCollection<KnownAzimuth> KnownAzimuthes
	{
		get { return knownAzimuthes; }
	}

private double m0 = 10; //中误差
private double fB;
private double FB; //角度闭合差的限差值
private double fx;
private double fy;
private double fs;

private double sumD;
private double K; // 1/K

//以下定义为绘图使用
private double minX;  //高斯坐标X的最小值xn
private double minY;  //高斯坐标Y的最小值yn
private double maxX; //高斯坐标X的最大值xm
private double maxY; //高斯坐标Y的最大值ym

private double maxVX; //屏幕坐标X的最大值
private double maxVY; //屏幕坐标Y的最大值

private double k;  //变换比例

private ObservableCollection<CtrPoint> ctrPoints =
new ObservableCollection<CtrPoint>();
public ObservableCollection<CtrPoint> CtrPoints
	{
		get { return ctrPoints; }
	}

private ObservableCollection<ObsValue> obsValues =
new ObservableCollection<ObsValue>();
public ObservableCollection<ObsValue> ObsValues
	{
		get { return obsValues; }
	}

/// <summary>
/// 正确的计算路线
/// </summary>
private List<ObsValue> route = new List<ObsValue>();

private bool isDirty = false;

public void ReadDataTextFile(string fileName)
{
using (System.IO.StreamReader sr = new System.IO.StreamReader(fileName))
{
string buffer;

//读入控制点数据
this.ctrPoints.Clear();
while (true)
{
		buffer = sr.ReadLine();
		if (string.IsNullOrEmpty(buffer)) break; //文件末尾或空行退出

		if (buffer[0] == '#') continue;

		string[] its = buffer.Split(new char[1] { ',' });
		if (its.Length != 4) continue; //不为四项控制点数据的继续,直到空行退出

		ctrPoints.Add(new CtrPoint(
		its[0].Trim(), // Name
		double.Parse(its[1]), //X
		double.Parse(its[2]), //Y
		double.Parse(its[3]), //H
		1)); //IsKnown
	}

//读入已知方位角信息:该节有可能不存在,也有可能有一条边,最多两条边
knownAzimuthes.Clear();
while (true)
{
		buffer = sr.ReadLine(); //由于是空行到此,所以继续往下读
		if (buffer == null) break; //文件末尾退出

		if (buffer == "" || buffer[0] == '#') continue; // 略过空行和注释行

		string[] its = buffer.Split(new char[1] { ',' });
		if (its.Length != 3) break; //数据项不为3,可能是角度观测值,退出当前

		KnownAzimuth ka = new KnownAzimuth();

		string ptName = its[0].Trim();
		ka.startPnt = GetCtrPoint(ptName);
		if (ka.startPnt == null)
		{
				ka.startPnt = new CtrPoint(ptName, 0, 0, 0, 0);//非已知点
				this.ctrPoints.Add(ka.startPnt);
			}

		ptName = its[1].Trim();
		ka.endPnt = GetCtrPoint(ptName);
		if (ka.endPnt == null)
		{
				ka.endPnt = new CtrPoint(ptName, 0, 0, 0, 0);//非已知点
				this.ctrPoints.Add(ka.endPnt);
			}

		ka.azimuth = ZXY.SMath.DMS2RAD(double.Parse(its[2]));
		knownAzimuthes.Add(ka);
	}

//读入观测值数据
this.obsValues.Clear();
while (true)
{
//此处可能由上不是3项数据的数据行退出,也有可能是文件末尾到此
//所以得先处理数据,后再读文本数据,否则,容易丢失数据
if (buffer == null) break; //文件末尾到此,继续退出
if (buffer == "" || buffer[0] == '#')//空行或正常的注释略过
{
buffer = sr.ReadLine();
continue;
}

string[] its = buffer.Split(new char[1] { ',' }); //进入正常的数据处理流程
if (its.Length == 5)
{
		string ptName = its[0].Trim();
		CtrPoint stnPnt = GetCtrPoint(ptName);
		if (stnPnt == null)
		{
				stnPnt = new CtrPoint(ptName, 0, 0, 0, 0);//非已知点
				this.ctrPoints.Add(stnPnt);
			}

		ptName = its[1].Trim();
		CtrPoint startPnt = GetCtrPoint(ptName);
		if (startPnt == null)
		{
				startPnt = new CtrPoint(ptName, 0, 0, 0, 0);//非已知点
				this.ctrPoints.Add(startPnt);
			}

		ptName = its[2].Trim();
		CtrPoint endPnt = GetCtrPoint(ptName);
		if (endPnt == null)
		{
				endPnt = new CtrPoint(ptName, 0, 0, 0, 0);//非已知点
				this.ctrPoints.Add(endPnt);
			}

		obsValues.Add(new ObsValue(
		stnPnt, startPnt, endPnt,
		double.Parse(its[3]), //AngleValue
		double.Parse(its[4]))); //Distance
	}

buffer = sr.ReadLine();
}
}
}

public void WriteDataTextFile(string fileName)
{
		using (System.IO.StreamWriter sw = new System.IO.StreamWriter(fileName))
		{
				sw.WriteLine("# Name, X, Y, Z");
				foreach (var pt in this.ctrPoints)
				{
						if(pt.IsKnown == 1) sw.WriteLine( pt );
					}

				sw.WriteLine();
				sw.WriteLine("# StartPnt, EndPnt, Azimuth");
				foreach (var az in this.knownAzimuthes)
				{
						sw.WriteLine( az );
					}

				sw.WriteLine();
				sw.WriteLine("# Station, StartPnt, EndPnt, Angle, Distance");
				foreach (var obs in this.ObsValues)
				{
						sw.WriteLine( obs );
					}
			}
	}


private ObsValue SearchStartObsValue()
{
		/**
		* 搜索起始边,首先:搜索直接给定的坐标方位角
		* 其次:上述搜索不成立的情况,搜索:测站点与后视点均为已知点的观测值
		* 再次:反向搜索:测站点与后视点均为已知点的观测值
		* */
		ObsValue obs = null;

		if (this.knownAzimuthes.Count > 0)
		{
				foreach (var azi in this.knownAzimuthes)
				{
						if (azi.endPnt.IsKnown == 1)
						{
								foreach (var it in obsValues)
								{
										if (it.StartPnt == azi.startPnt && it.StnPnt == azi.endPnt)
										{
												obs = it;
												return obs;
											}
									}
							}
					}
			}
		else if (this.knownAzimuthes.Count == 0)
		{
				foreach (var it in obsValues)
				{
						double az = 0;

						if (it.StartPnt.IsKnown == 1 && it.StnPnt.IsKnown == 1)
						{
								az = ZXY.SMath.Azimuth(it.StartPnt.X, it.StartPnt.Y, it.StnPnt.X, it.StnPnt.Y);
								this.knownAzimuthes.Add(new KnownAzimuth(it.StartPnt, it.StnPnt, az));
								obs = it;
							}

						if (it.StnPnt.IsKnown == 1 && it.EndPnt.IsKnown == 1)
						{
								az = ZXY.SMath.Azimuth(it.StnPnt.X, it.StnPnt.Y, it.EndPnt.X, it.EndPnt.Y);
								this.knownAzimuthes.Add(new KnownAzimuth(it.StnPnt, it.EndPnt, az));
							}
					}
			}
		return obs;
	}


/// <summary>
/// 递归搜索观测值obs0的下一条边
/// </summary>
/// <param name="obs0">当前观测值</param>
/// <returns>1:附合导线, -1:不构成附合导线</returns>
private int SearchObsValue(ObsValue obs0)
{
//传进来的第一条边应为起始边
ObsValue obsi = null;

foreach (var it in obsValues)
{
if (it == obs0) continue;

if (obs0.StnPnt == it.StartPnt && obs0.EndPnt == it.StnPnt)  //满足条件的下一条边
{
obsi = it;
break;
}
}

if (obsi == null) return -1;//没找到这样的边

this.route.Add(obsi);
if (obsi.StnPnt.IsKnown == 1 && obsi.EndPnt.IsKnown == 1) //附合到另一条已知边了,退出
{
return 1;
}
else
SearchObsValue(obsi); //递归继续寻找下一条这样的边

return 0;
}


/// <summary>
/// 搜索正确的计算路线
/// </summary>
/// <returns>是否成功</returns>
public bool SearchCalRoute()
{
		ObsValue obs0 = SearchStartObsValue();
		if (obs0 == null) return false;

		route.Clear(); //清空搜索线路

		route.Add(obs0);
		SearchObsValue(obs0);
		return true;
	}

/// <summary>
/// 简易平差
/// </summary>
/// <returns>0:成功</returns>
public int Adjust()
{
		/*
		1. 求起始边: 后视->测站 的方位角
		求末边:测站->前视 的方位角
		2. 计算角度闭合差fB,FB
		3. 改正角度值
		4. 推算各边坐标方位角
		5. 计算各边的坐标增量
		6.计算fx, fy, fs, 1/K
		7.计算改正后的坐标增量
		8.计算各点的坐标值
		*/
		if (this.obsValues.Count == 0) return -1; //观测值为空

		if (SearchCalRoute() == false) return -2; //搜索不到正确的附合路线

		double az0 = this.knownAzimuthes[0].azimuth;
		CtrPoint startPnt0 = this.knownAzimuthes[0].startPnt;
		CtrPoint stnPnt0 = this.knownAzimuthes[0].endPnt;

		double azn = this.knownAzimuthes[1].azimuth;
		CtrPoint stnPntn = this.knownAzimuthes[1].startPnt;
		CtrPoint endPntn = this.knownAzimuthes[1].endPnt;

		double azi = az0;
		double n = route.Count;
		foreach (var obs in route) //foreach (var obs in this.ObsValues)
		{
				obs.Azimuth = azi + obs.AngleValue + Math.PI;
				if (obs.Azimuth >= 2 * Math.PI) obs.Azimuth -= 2 * Math.PI;
				if (obs.Azimuth < 0) obs.Azimuth += 2 * Math.PI;

				azi = obs.Azimuth;
			}

		fB = azi - azn; //单位:弧度
		FB = m0 * 2 * Math.Sqrt(n); //单位:秒

		//改正角度,推算各边改正后的方位角
		azi = az0;
		double vi = -fB / n;
		foreach (var obs in route) //foreach (var obs in this.ObsValues)
		{
				obs.VB = vi;
				obs.AngleV = obs.AngleValue + obs.VB;
				obs.Azimuth = azi + obs.AngleV + Math.PI;
				if (obs.Azimuth >= 2 * Math.PI) obs.Azimuth -= 2 * Math.PI;
				if (obs.Azimuth < 0) obs.Azimuth += 2 * Math.PI;
				azi = obs.Azimuth;
			}

		//计算各边的坐标增量
		double sumDX = 0, sumDY = 0;
		sumD = 0;
		foreach (var obs in route) //foreach (var obs in this.ObsValues)
		{
				if (obs.Distance <= 0) continue;

				obs.DX = obs.Distance * Math.Cos(obs.Azimuth);
				obs.DY = obs.Distance * Math.Sin(obs.Azimuth);
				sumDX += obs.DX;
				sumDY += obs.DY;
				sumD += obs.Distance;
			}
		fx = stnPnt0.X + sumDX - stnPntn.X;
		fy = stnPnt0.Y + sumDY - stnPntn.Y;
		fs = Math.Sqrt(fx * fx + fy * fy);
		K = sumD / fs;

		//改正坐标增量,计算各点坐标
		foreach (var obs in route) //foreach (var obs in this.ObsValues)
		{
				if (obs.Distance <= 0) continue;

				obs.VDX = -fx / sumD * obs.Distance;
				obs.VDY = -fy / sumD * obs.Distance;

				obs.EndPnt.X = obs.StnPnt.X + obs.DX + obs.VDX;
				obs.EndPnt.Y = obs.StnPnt.Y + obs.DY + obs.VDY;
			}

		return 0;
	}

private CtrPoint GetCtrPoint(string pointName)
{
		foreach (var pt in this.ctrPoints)
		{
				if (pt.Name == pointName)
				return pt;
			}

		return null;
	}

public void OnDraw(DrawingCanvas canvas)
{
		if (this.ctrPoints.Count < 1) return;

		GetGaussXySize();

		maxVX = canvas.ActualWidth-20;
		maxVY = canvas.ActualHeight;

		double kx = maxVX / (maxY - minY);
		double ky = maxVY / (maxX - minX);
		k = kx <= ky ? kx : ky;

		canvas.ClearAll(); //先清除屏幕

		//画观测值
		double x0, y0, x1, y1, x2, y2; //画直线的两个端点
		foreach (var it in this.obsValues)
		{
				if (it.StnPnt.X <= 0 && it.StnPnt.Y <= 0) continue; //略过坐标为0的点
				GaussXyToViewXy(it.StnPnt.X, it.StnPnt.Y, out x0, out y0);

				if (it.StartPnt.X > 0 && it.StartPnt.Y > 0)
				{
						GaussXyToViewXy(it.StartPnt.X, it.StartPnt.Y, out x1, out y1);
						canvas.DrawLine(x1, y1, x0, y0, Brushes.Black, 1);
					}

				if (it.EndPnt.X > 0 && it.EndPnt.Y > 0)
				{
						GaussXyToViewXy(it.EndPnt.X, it.EndPnt.Y, out x2, out y2);
						canvas.DrawLine(x0, y0, x2, y2, Brushes.Black, 1);
					}
			}

		//再画控制点
		foreach (var pt in this.ctrPoints)
		{
				if (pt.X <= 0 && pt.Y <= 0) continue; //排除坐标为0的点

				GaussXyToViewXy(pt.X, pt.Y, out x0, out y0);
				if(pt.IsKnown == 1)
				canvas.DrawKnCtrPnt(x0, y0, Brushes.Black, 1);
				else
				canvas.DrawCtrPnt(x0, y0, Brushes.Black, 1);
				canvas.DrawText(pt.Name, x0 + 10, y0 - 7);
			}
	}

private void GaussXyToViewXy(double xt, double yt, out double xp, out double yp)
{
		//xp = x0 + kx(yt - yn);
		//yp = y1 - (y0 + ky * (xt - xn));
		// x0 = y0 =0 且 kx = ky =k, 故以上公式简化为:

		xp = 5 + k * (yt - minY); //x0 = 5;
		yp = maxVY - (5 + k * (xt - minX)); //y0=5;
	}

private void GetGaussXySize()
{
minX = this.ctrPoints[0].X; minY = this.ctrPoints[0].Y;
maxX = this.ctrPoints[0].X; maxY = this.ctrPoints[0].Y;

for (int i = 1; i < this.ctrPoints.Count; i++) //如果只有一个点,由循环条件知,不会执行循环体
{
if (this.ctrPoints[i].X <= 0 && this.ctrPoints[i].Y <= 0) continue;

if (this.ctrPoints[i].X < minX) minX = this.ctrPoints[i].X;
if (this.ctrPoints[i].Y < minY) minY = this.ctrPoints[i].Y;

if (this.ctrPoints[i].X > maxX) maxX = this.ctrPoints[i].X;
if (this.ctrPoints[i].Y > maxY) maxY = this.ctrPoints[i].Y;
}

//针对一个点或点范围较小的情况,进行范围扩展
if (minX + 10 > maxX) { maxX = minX + 10; minX = maxX - 20; }
if (minY + 10 > maxY) { maxY = minY + 10; minY = maxY - 20; }
}
}
}

using System.Globalization;
using System.Windows;
using System.Windows.Media;

namespace SurApp.models
	{
		public class DrawingCanvas : System.Windows.Controls.Canvas
			{
				private VisualCollection visuals;

				public DrawingCanvas()
				{
						visuals = new VisualCollection(this);
					}

				//获取Visual的个数
				protected override int VisualChildrenCount
					{
						get { return visuals.Count; }
					}

				//获取Visual
				protected override Visual GetVisualChild(int index)
				{
						return visuals[index];
					}

				//添加Visual
				public void AddVisual(Visual visualObject)
				{
						visuals.Add( visualObject );
					}

				//删除Visual
				public void RemoveVisual(Visual visualObject)
				{
						base.RemoveLogicalChild(visualObject);
					}

				//命中测试
				public DrawingVisual GetVisual(System.Windows.Point point)
				{
						HitTestResult hitResult = VisualTreeHelper.HitTest(this, point);
						return hitResult.VisualHit as DrawingVisual;
					}

				public void ClearAll()
				{
						this.visuals.Clear();
					}


				//使用DrawVisual画Polyline
				public void DrawLine(double x0, double y0, double x1, double y1,  Brush color, double thinkness)
				{
						DrawingVisual visualLine = new DrawingVisual();
						DrawingContext dc = visualLine.RenderOpen();
						Pen pen = new Pen(color, thinkness);
						pen.Freeze();  //冻结画笔,这样能加快绘图速度
						dc.DrawLine(pen, new Point(x0, y0),  new Point(x1, y1) );

						dc.Close();
						visuals.Add(visualLine);
					}

				public void DrawText(string text, double x, double y)
				{
						DrawingVisual visualText = new DrawingVisual();
						DrawingContext dc = visualText.RenderOpen();
						Typeface tp = new Typeface(new FontFamily("宋体"), FontStyles.Normal, FontWeights.Normal, FontStretches.Normal);
						FormattedText ft = new FormattedText(text, CultureInfo.CurrentCulture,
						FlowDirection.LeftToRight, tp, 12, Brushes.Black);
						dc.DrawText(ft, new Point(x, y) );
						dc.Close();
						visuals.Add( visualText );
					}

				//使用DrawVisual画Circle,用作控制点
				public void DrawCtrPnt(double x, double y,  Brush color, double thinkness)
				{
						DrawingVisual visualCircle = new DrawingVisual();
						DrawingContext dc = visualCircle.RenderOpen();
						Pen pen = new Pen(color, thinkness);
						pen.Freeze();  //冻结画笔,这样能加快绘图速度
						dc.DrawEllipse(Brushes.LemonChiffon,  pen,  new Point(x, y),  5, 5);
						dc.Close();
						visuals.Add(visualCircle);
					}

				//使用DrawVisual画Circle,用作控制点
				public void DrawKnCtrPnt(double x, double y, Brush color, double thinkness)
				{
						DrawingVisual visualCircle = new DrawingVisual();
						DrawingContext dc = visualCircle.RenderOpen();
						Pen pen = new Pen(color, thinkness);
						pen.Freeze();  //冻结画笔,这样能加快绘图速度
						dc.DrawEllipse(Brushes.Black, pen, new Point(x, y), 1, 1);
						dc.DrawLine(pen, new Point(x-5, y+2.9), new Point(x+5, y+2.9));
						dc.DrawLine(pen, new Point(x + 5, y + 2.9), new Point(x, y - 5.8));
						dc.DrawLine(pen, new Point(x, y - 5.8), new Point(x - 5, y + 2.9));
						dc.Close();
						visuals.Add(visualCircle);
					}
			}
	}

 \end{lstlisting}
