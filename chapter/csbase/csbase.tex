%!TEX root = ../../clcxsj.tex

\chapter{ \cs 语言基础}

\cs 程序设计语言是类C语言,基本语法与C基本相同, 本章主要讲解 \cs 中不同于 C 的基础语法。

\section{ \cs 中的数据类型}
 为保持 \cs 语言的高效性,\cs 中的数据类型分为两类:值类型 (value type) 和引用类型 (reference type)。
值类型的变量直接包含它们的数据,也就是直接在栈中保存变量的值, 变量的值自动生成、自动释放;
而引用类型的变量存储对数据的引用,即在堆中生成变量的值,在栈中保存堆中值的地址,在堆中生成的变量不自动释放,靠 .Net 架构回收。

\subsection{值类型}
 \cs  的值类型进一步划分为简单类型 (simple type)、枚举 (enum)、结构(struct)
和可以为 null 的值类型 (nullable type)。

对于值类型,每个变量都有它们自己的数据副本(除 ref 和 out 参数变量外),
因此对一个变量的操作不可能影响另一个变量。

\begin{tabular}{|l|l|}
\hline
类别   & 说明 \\
\hline
简单类型   & 有符号整型:sbyte、short、int、long \\
          & 无符号整型:byte、ushort、uint、ulong \\
          & Unicode 字符:char \\
          & IEEE 浮点:float、double \\
          & 高精度小数型:decimal \\
          & 布尔:bool \\
\hline
结构                & struct S {...} 形式的用户定义的类型 \\
枚举                & enum E {...} 形式的用户定义的类型   \\
可以为 null 的值类型 & 其他所有具有 null 值的值类型的扩展   \\
\hline
\end{tabular}


\subsection{引用类型}
 \cs  的引用类型进一步划分为类 (class)、接口(interface)、
数组 (array) 和委托(delegate)。
对于引用类型,两个变量可能引用同一个对象,因此对一个变量的操作可能影响另一个变量所引用的对象。

\begin{tabular}{|l|l|}
\hline
类别          & 说明 \\
\hline
类class       &  所有其他类型的最终基类:object \\
              &  Unicode 字符串:string \\
              &  class C {...} 形式的用户定义的类型 \\
\hline
接口interface     & interface I {...} 形式的用户定义的类型  \\
数组              & 一维和多维数组,例如 int[] 和 int[,]  \\
委托 delegate     & delegate int D(...) 形式的用户定义的类型  \\
\hline
\end{tabular}

 \cs 中的 string不是值类型数据,而是引用类型数据。

\subsection{自定义类型}
 \cs  程序使用类型声明 (type declaration) 创建新类型。类型声明指定新类型的名称和成员。
在  \cs  类型分类中,有五类是用户可定义的:类(class)、结构(struct)、
接口(interface)、枚举(enum) 和委托(delegate)。

注:\cs 7.0 提供了元组(tuple), \cs 9.0 提供了记录(record)。


类是定义了一个包含数据成员(字段)和函数成员(方法、属性等)的数据结构,类支持单一继承和多态。

结构与类相似,表示一个带有数据成员和函数成员的结构。
但是,与类不同,结构是值类型,不需要堆分配。
结构类型不支持用户指定的继承,所有结构类型都隐式地从类 System.Object 继承。

接口定义了一个协定,是一个公共函数成员的命名集。
实现某个接口的类或结构必须提供该接口的函数成员的实现。
一个接口可以从多个接口继承,而一个类或结构可以实现多个接口。

委托表示对具有特定参数列表和返回类型的方法的引用。
通过委托,我们能够将方法作为实体赋值给变量和作为参数传递。
委托类似于在其他某些语言中的函数指针的概念,但是与函数指针不同,委托是面向对象的,并且是类型安全的。

类、结构、接口和委托都支持泛型,因此可以通过其他类型将其参数化。

枚举是具有命名常量的独特的类型。每种枚举类型都具有一个基础类型,
该基础类型必须是八种整型之一。枚举类型的值集和它的基础类型的值集相同。

 \cs  支持由任何类型组成的一维和多维数组。与以上列出的类型不同,数组类型不必声明就可以使用。
实际上,数组类型是通过在某个类型名后加一对方括号来构造的。
例如,int[] 是一维 int 数组,int[,] 是二维 int 数组,int[][] 是一维 int 数组的一维数组。

可以为 null 的类型也不必声明就可以使用。对于每个不可以为 null 的值类型 T,
都有一个相应的可以为 null 的类型 T?,该类型可以容纳附加值 null。
例如,int? 类型可以容纳任何 32 位整数或 null 值。

\subsection{装箱与拆箱}
 \cs  的类型系统是统一的,因此任何类型的值都可以按对象处理。 \cs  中的每个类型直接或间接地从 object 类类型派生,
而 object 是所有类型的根基类。引用类型的值都被视为 object 类型,被简单地当作对象来处理。
值类型的值则通过对其执行装箱和拆箱操作按对象处理。下面的示例将 int 值转换为 object,然后又转换回 int。

\begin{lstlisting}
using System;
class Test
{
    static void Main()
    {
        int i = 123;
        object o = i;      // Boxing
        int j = (int)o;    // Unboxing
    }
}
\end{lstlisting}

当将值类型的值转换为类object 时,将分配一个对象实例(也称为“箱子”)以包含该值,
并将值复制到该箱子中。反过来,当将一个 object 引用强制转换为值类型时,
将检查所引用的对象是否含有正确的值类型,如果有,则将箱子中的值复制出来。


\subsection{延伸阅读值类型}

实际上值类型和引用类型都是从System.Object继承的,System.Object类是 \cs 中所有类的基类。
所有值类型都隐式继承自类 System.ValueType,而类System.ValueType又继承自类 System.Object(别名object),
System.ValueType 用更合适的值类型实现重写了Object 中的虚方法。 

但任何类型都不能直接从值类型System.ValueType派生。

这样就有一个问题,值类型与引用类型都是从System.Object继承的,怎么object类型与int类型就不一样呢?

这里大致可以这样理解,.Net是动态加载运行 \cs 代码的,CLR在运行时把值类型装载到栈,把引用类型装载到堆,
栈中只提供一个引用指向堆,即引用。

\subsection{\cs 新版本中的类型}

\begin{itemize}
    \item 元组(tuple)
    
    用于将多个数据元素分组成一个轻型数据结构。
    如:
\begin{lstlisting}
(double, int) t1 = (4.5, 3);
Console.WriteLine($"Tuple with elements {t1.Item1} and {t1.Item2}.");
// Output:
// Tuple with elements 4.5 and 3.
(double Sum, int Count) t2 = (4.5, 3);
Console.WriteLine($"Sum of {t2.Count} elements is {t2.Sum}.");
// Output:
// Sum of 3 elements is 4.5.
\end{lstlisting}

    元组类型是值类型,元组元素是公共字段。元组字段的默认名称为 Item1、Item2、Item3 等。

    \item 记录(record)
    
    record类型是一种用record关键字声明的新的引用类型,与类不同的是,
    它是基于值相等而不是唯一的标识符——对象引用。
    它有着引用类型的支持大对象、继承、多态等特性,也有着结构的基于值相等的特性。
    可以说有着class和struct两者的优势,在一些情况下可以用以替代class和struct。

    如以下代码就定义了一个类Point:
\begin{lstlisting}
public record Point(string Name, double X, double Y, double Z);
\end{lstlisting}

\end{itemize}


\subsection{变量的定义}

\cs 中变量是前置定义的,需先指定类型,然后是变量名,最后是其值。
如: int i = 0;

变量可以显示指定类型,也可以隐式定义,让编译器根据变量的值进行推定,如:
var i = 0;

值类型数据可以在声明时赋初始值,引用类型变量在定义时需用new生成实例对象或
将其它实例对象的引用赋值给它,或直接赋值为null。


\section{表达式与语句}

\subsection{表达式}
表达式由操作数 (operand) 和运算符 (operator) 构成。表达式的运算符指示对操作数适用什么样的运算。
运算符的示例包括+、-、*、/ 和 new。操作数的示例包括文本、字段、局部变量和表达式。

\begin{tabular}{|l|l|}
\hline
类别            & 说明 \\
\hline
x.m            &          成员访问  \\
x(...)         &           方法和委托调用  \\
x[...]         &           数组和索引器访问  \\
x++            &        后增量  \\
x--            &           后减量  \\
new T(...)     &       对象和委托创建  \\
new T(...){...}&      使用初始值设定项创建对象  \\
new {...}      &         匿名对象初始值设定项  \\
new T[...]     &        数组创建  \\
typeof(T)      &      获取 T 的 System.Type 对象  \\
checked(x)     &    在 checked 上下文中计算表达式  \\
unchecked(x)   & 在 unchecked 上下文中计算表达式  \\
default(T)     &       获取类型 T 的默认值  \\
delegate {...} &      匿名函数(匿名方法)  \\
(T)x           &             将 x 显式转换为类型 T  \\
await x        &         异步等待 x 完成  \\
x is T         &            如果 x 为 T,则返回 true,否则返回 false  \\
x as T         &           返回转换为类型 T 的 x,如果 x 不是 T 则返回 null  \\
(T x) => y     &       匿名函数(lambda 表达式)  \\
\hline
\end{tabular}

其它的与 C 或 \cpp 语言基本相同。


\subsection{ 语句 }
程序的操作是使用语句 (statement) 来表示与实现的。

\cs 中常见的语句有:
\begin{itemize}
\item 声明语句 (declaration statement) 

    用于声明局部变量和常量。

\item 表达式语句 (expression statement) 

    用于对表达式求值。

    可用作语句的表达式包括方法调用、使用 new 运算符的对象分配、
使用 = 和复合赋值运算符的赋值、使用 ++ 和 -- 运算符的增量和减量运算以及 await 表达式。

\item 选择语句 (selection statement) 

用于根据表达式的值从若干个给定的语句中选择一个来执行。
这一组中有 if else、 switch case、 ? : 、 ??语句。

??表示如果左值为null,则用右值进行计算,如下代码所示:
\begin{lstlisting}
int? a = null;
int b = a ?? -1;
Console.WriteLine(b);  // output: -1
\end{lstlisting}

\item 迭代语句 (iteration statement) 

用于重复执行嵌入语句,常用语句有 while、do、for 和 foreach 语句。

\cs 中常用foreach语句进行循环,示例代码如下:

\begin{lstlisting}
static void Main(string[] args)
{
    foreach (string s in args)
    {
        Console.WriteLine(s);
    }
}
\end{lstlisting}


\item 跳转语句 (jump statement) 

用于转移控制,常用语句有 break、continue、goto、throw、return 和 yield 语句。

\item 异常捕捉

try...catch 语句用于捕获在块的执行期间发生的异常,try...finally 语句用于指定终止代码,不管是否发生异常,该代码都始终要执行。
    
\begin{lstlisting}
static double Divide(double x, double y)
{
    if (y == 0) throw new DivideByZeroException();
    return x / y;
}

static void Main(string[] args)
{
    try
    {
        if (args.Length != 2)
        {
            throw new Exception("Two numbers required");
        }
        double x = double.Parse(args[0]);
        double y = double.Parse(args[1]);
        Console.WriteLine(Divide(x, y));
    }
    catch (Exception e)
    {
        Console.WriteLine(e.Message);
    }
    finally
    {
        Console.WriteLine(“Good bye!”);
    }
}
\end{lstlisting}

\item checked 语句和 unchecked 语句

用于控制整型算术运算和转换的溢出检查上下文。

\begin{lstlisting}
static void Main()
{
    int i = int.MaxValue;
    checked 
    {
        Console.WriteLine(i + 1);       // Exception
    }
    unchecked
    {
        Console.WriteLine(i + 1);       // Overflow
    }
}
\end{lstlisting}

\item lock 语句

用于获取某个给定对象的互斥锁,执行一个语句,然后释放该锁。

\begin{lstlisting}
class Account
{
    decimal balance;
    public void Withdraw(decimal amount)
    {
        lock (this)
        {
            if (amount > balance)
            {
                throw new Exception("Insufficient funds");
            }
            balance -= amount;
        }
    }
}
\end{lstlisting}

\item using 语句

用于获得一个资源,执行一个语句,然后释放该资源。

\begin{lstlisting}
static void Main()
{
    using (TextWriter w = File.CreateText("test.txt"))
    {
        w.WriteLine("Line one");
        w.WriteLine("Line two");
        w.WriteLine("Line three");
    }
}
\end{lstlisting}


\end{itemize}


\section{ \cs 特殊语法}

\subsection{泛型}
在定义类时,在类名后用尖括号括起来的类型参数列表,称之为泛型。如:
\begin{lstlisting}
public class Pair<T1, T2>
{
    public T1 first;
    public T2 second;
}
\end{lstlisting}
代码中的T1与T2为类型参数。

泛型是现代编程语言非常重要的语法。

\subsection{集合}
数组Array、列表List、词典Dictionary、栈Stack、队列Queue 
是程序设计中常用的数据结构,在 \cs 中均以类库形式提供,
在\cs 程序设计中可以直接使用。

在使用这些集合类时,应尽量使用泛型集合,如动态数组:List<int>

\subsection{委托}
委托类似于C、\cpp 中的函数指针。语法为:

访问修饰符 delegate 返回类型 委托名(参数序列);

从语法上可以看出,delegate之后实际上就是函数的定义。

.Net中预定义了Func和Action委托。

Func委托代表有返回值的委托,参数最多有16个,最后一个参数必须是返回参数TResult,语法形式为:

public delegate TResult Func<in T1, ..., in T15, out TResult>(T1 arg1, ..., T15 arg15);

Action委托代表无有返回值的委托,参数最多有16个,语法形式为:

public delegate void Action<in T1,...,in T16>(T1 arg1, ..., T16 arg2);

\subsection{Lambda语句}
Lambda表达式主要用于简化委托的编写形式,是一种可用于创建委托或表达式树类型的匿名函数。语法为:

(输入参数列表) => \{表达式或语句块\}

Lambda表达式实质是一个匿名函数,如下代码所示:
\begin{lstlisting}
List<int> list = new List<int>(){1,2,3,4,5,6,7,8,10,11,12,13,14,15};
var q = list.Were( x => x%2 == 0);
\end{lstlisting}
list.Where函数中调用的就是Lambda表达式。

\subsection{迭代器}

迭代器用于逐步迭代集合(如列表和数组)。

迭代器方法或 get 访问器可对集合执行自定义迭代。 迭代器方法使用 yield return 语句返回元素,每次返回一个。 到达 yield return 语句时,会记住当前在代码中的位置。 下次调用迭代器函数时,将从该位置重新开始执行。

通过 foreach 语句或 LINQ 查询从客户端代码中使用迭代器。

如下所示代码,将输出 3 5 8

\begin{lstlisting}
using System;

namespace ConsoleApp1
{
    class Program
    {
        static void Main(string[] args)
        {
            foreach (int number in SomeNumbers())
            {
                Console.Write(number.ToString() + " ");
            }
            // Output: 3 5 8
            Console.ReadKey();

        }

        public static System.Collections.IEnumerable SomeNumbers()
        {
            yield return 3;
            yield return 5;
            yield return 8;
        }
    }
}
\end{lstlisting}

迭代器方法或 get 访问器的返回类型可以是 IEnumerable、IEnumerable<T>、IEnumerator 或 IEnumerator<T>。

可以使用 yield break 语句来终止迭代。

yield语句

\begin{lstlisting}
static IEnumerable<int> Range(int from, int to) {
    for (int i = from; i < to; i++) {
        yield return i;
    }
    yield break;
}
static void Main()
{
    foreach (int x in Range(-10,10)) {
        Console.WriteLine(x);
    }
}
\end{lstlisting}



\section{小结}
这一章我们学习了 \cs 基本语法:数据与语句,数据分为数值类型与引用类型,
数据定义时可以使用var关键字让编译器推断变量的数据类型,还可以使用泛型。

\cs 语句相对于C、\cpp 而言,出现了一些新的如lambda语句、迭代器等,
这些是许多现代语言中必不可少的语言特征,
我们将在后续课程中将其继续学习与应用。

\section{上机实验}
\begin{enumerate}
\item 冒泡排序(Bubble Sort)

冒泡排序是一种较简单的排序算法。它重复地走访过要排序的元素列,
依次比较两个相邻的元素,如果顺序错误就把他们交换过来。
走访元素的工作是重复地进行直到没有相邻元素需要交换,也就是说该元素列已经排序完成。
这个算法的名字由来是因为越小的元素会经由交换慢慢“浮”到数列的顶端(升序或降序排列),
就如同碳酸饮料中二氧化碳的气泡最终会上浮到顶端一样,故名“冒泡排序”。

原理如下:

比较相邻的元素。如果第一个比第二个大,就交换他们两个。

针对所有的元素重复以上的步骤,除了最后一个。

持续每次对越来越少的元素重复上面的步骤,直到没有任何一对数字需要比较;

\item 快速排序(QuickSort)

基本思想:通过一趟排序将待排记录分隔成独立的两部分,其中一部分记录的关键字均比另一部分的关键字小,则可分别对这两部分记录继续进行排序,以达到整个序列有序。

算法描述

    快速排序使用分治法来把一个串(list)分为两个子串(sub-lists)。具体算法描述如下:

    从数列中挑出一个元素,称为 “基准”(pivot);

    重新排序数列,所有元素比基准值小的摆放在基准前面,所有元素比基准值大的摆在基准的后面(相同的数可以到任一边)。在这个分区退出之后,该基准就处于数列的中间位置。这个称为分区(partition)操作;

    递归地(recursive)把小于基准值元素的子数列和大于基准值元素的子数列排序。

\end{enumerate}

