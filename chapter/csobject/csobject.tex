%!TEX root = ../../clcxsj.tex

\chapter{\cs 面向对象特性}

类(class) 是面向对象程序设计的基础与核心,封装、继承、多态(polymorphism)是面向对象程序设计的三大特征。

封装把客观事物封装成抽象的类,在类中把自己的数据和方法只让可信的类或者对象操作,对不可信的类或对象进行信息隐藏。

也就是应尽可能的把类成员变量的访问权限定为 private 或 protected ,通过属性(Property)或方法向外开放;
类的方法/函数 应尽可能的定义为public;
需向下继承的类成员,尽可能可以定义为 protected;
项目内能访问的成员,尽可能定义为internal。

继承(Inheritance)是指这样一种能力:它可以使用现有类的所有功能,并在无需重新编写原来的类的情况下对这些功能进行扩展。

继承符合开闭原则。

通过继承创建的新类称为“子类”或“派生类”, 被继承的类称为“基类”、“父类”或“超类”。
继承的过程,是从一般到特殊的过程。

多态(polymorphism)是指将子对象赋值给父对象,父对象就可以根据当前赋值给它的子对象的特性以不同的方式运作。


\section{类与对象}
类是 \cs 中最基础最重要的类型,是一种将数据(字段)与对数据的操作(方法、函数以及属性等)组合在一起的数据结构。

严格地说,类分为实例类与静态类,实例类为动态创建类的实例 (instance) 提供定义,
实例也称为对象 (object)。也可以说类是定义,对象是类的实例。
静态类不能生成类的实例,在此声明,若非特别语境下,以后本书中所指的类均为实例类。

在 \cs 中类支持单继承 (inheritance),通过类继承产生派生类 (derived class), 
从而扩展和专用化基类 (base class) 。

 \cs 中所有的类均从 System.Object 继承,如果一个类只从System.Object 类继承,
 则可以将其省略。

 \subsection{类的定义}

类的关键字为class,定义一个新类的方式如下:

\begin{lstlisting}
[public/internal] class ClassName [: BaseClass, Interface1,...,InterfaceN]
{
    ......
}
\end{lstlisting}

先指定类的修饰符,class关键字后是类名称(类名称首字母一般大写),接着是要继承的基类
与需实现的接口。
之后是类的主体,它由一组位于一对大括号 ``\{'' 和 ``\}'' 之间的成员声明组成。

由于测绘行业的基本任务是提供位置服务的,点位是我们经常使用的概念,我们就对它来定义一个类 Point:

\begin{lstlisting}
public class Point : System.Object
{
    ......
}
\end{lstlisting}

上面是测量点的简单定义,关键字 class 前的 public 为访问限定符,表示访问不受限制。 
也可以为internal,表示该类仅限于当前程序集访问,外部程序集不能访问。
如果在class前不加访问修饰符,默认的访问权限为internal。

点名后的 ``: System.Object'' 表示 Point类从 System.Object 类(System为命名空间)继承。
\cs 中 System.Object 类为所有类的 根基类,如果不写类的基类,则默认从System.Object继承,
上例中 ``: System.Object'' 可以省略。

上面的代码可以简写为:
\begin{lstlisting}
public class Point
{
    ......
}
\end{lstlisting}

%%%%%%%%%%%%%%%%%%%%%%%%%%%%%%%%%%%%%%%%%%%%%%%%%%%%%%%%%%%%%%%%%%%%%%
\subsection{类的实例对象}

在 \cs 中用 new 操作符生成类的实例对象。该运算符为新的实例分配内存、
调用构造函数初始化实例,并返回对该实例的引用。

下面的语句生成两个 Point 实例对象,并将对这两个对象的引用存储在两个变量p1、p2中:

\begin{lstlisting}
{
    ......
    Point p1 = new Point();
    Point p2 = new Point();
    ......
}
\end{lstlisting}

当程序执行到这段代码后边的 ``\}'' 时,将自动释放变量 p1、p2,
p1与p2所指向的内存,将由 .Net 系统自动收回。

\subsection{类的字段(Filed)}
一个类通常由两部分组成:数据与对数据的操作。字段是与类或类的实例相关的变量,
用于存储类的数据。

如上面的点Point类,我们可以为其定义数据成员如下:

\begin{lstlisting}
public class Point
{
    public string name;
    public string code;
    public double x;
    public double y;
    public double z;
}
\end{lstlisting}

在定义点时,点的位置属性(x,y,z)很容易理解,但很容易忽略点的点名与编码两个属性,
其实,测绘中往往用点名指代某个位置,点也有控制点与一般点之分,需要用编码区分点的类型,
所以点名于编码这两个字段也就很有必要。

\subsection{类字段的可访问性}

从以上点Point类的定义看出,每个字段的定义前都有public,代表着对字段的访问权限。

类的每个成员都可以关联可访问性,以控制访问该类成员的权限,默认的可访问性为private。

有六种可能的可访问性形式,如下表所示:

\begin{tabular}{|l|l|}
\hline
可访问性                 &   含义  \\
\hline
public                  &  访问不受限制   \\
protected               &  访问仅限于该类及该类的派生类  \\
internal                &  访问仅限于当前程序集  \\
protected internal      &  访问仅限于当前程序集或该类的派生类  \\
private                 &  访问仅限于包含类  \\
private protected       &  访问限于包含类或当前程序集中派生自包含类的类型。 \\
\hline
\end{tabular}

一个类可以访问自己的属性,可以通过this关键字进行引用访问,在不引起混淆的情况下,可以省略this。
this关键字指类的实例本身。

在Point类的定义 public 访问权限的数据成员后,就可以按照``实例.数据成员''进行引用了,
如下代码所示:

\begin{lstlisting}
{
    ......
    Point p1 = new Point();
    p1.name = "p1";
    p1.x = 100.0;
    p1.y = 100.0;
    p1.z = 410.0;

    Point p2 = new Point();
    p2.name = "p2";
    p2.x = 200.0;
    p2.y = 200.0;
    p2.z = 420.0;
    ......
}
\end{lstlisting}

\subsection{类的构造函数}

从上面的p1、p2的引用语句看,实际上是在给p1、p2两个点赋初值。
如果每个点都这样做的话,十分不方便。初始化实例对象更好的方法是使用构造函数。

\textbf{构造函数是函数名称与类名完全相同,且没有返回类型(即使返回类型为void也不行)的函数,
它的作用是做类的初始化工作(主要是初始化类的数据)。}

定义与使用Point类的代码如下:
\begin{lstlisting}
//在Point.cs文件中定义
public class Point : System.Object
{
    public string name;
    public string code;
    public double x;
    public double y;
    public double z;

    public Point(string name, string code, double x, double y, double z)
    {
        this.name = name;
        this.code = code;
        this.x = x;
        this.y = y;
        this.z = z;
    }
}

//在Program.cs文件中Main()函数中使用
Point p1 = new Point("p1", "", 100.0, 100.0, 410.0);
Point p2 = new Point("p2", "", 200.0, 200.0, 420.0);
\end{lstlisting}

构造函数支持重载(overload)。所谓的\textbf{函数重载} 就是函数名称相同,但函数的参数类型或参数个数不一样。
请注意,函数重载没有关注函数的返回值,只关注了函数的参数。

没有参数的构造函数称为\textbf{默认构造函数}。如果类中不定义任何构造函数,系统会为我们生成一个默认构造函数,
一旦我们定义了一个构造函数,系统就不会为我们再生成这个默认构造函数。
如果仍然要使用这个默认的构造函数,需要显式定义。

为了方便Point类的初始化,我们为Point类定义多个构造函数如下:

\begin{lstlisting}
//在Point.cs文件中定义
public class Point : System.Object
{
    public string name;
    public string code;
    public double x;
    public double y;
    public double z;

    public Point() //默认构造函数
    {
        this.name = null;
        this.code = null;
        this.x = 0;
        this.y = 0;
        this.z = 0;
    }

    public Point(double x, double y)
    {
        this.name = null;
        this.code = null;
        this.x = x;
        this.y = y;
        this.z = 0;
    }

    public Point(double x, double y, double z)
    {
        this.name = null;
        this.code = null;
        this.x = x;
        this.y = y;
        this.z = z;
    }

    public Point(string name, double x, double y)
    {
        this.name = name;
        this.code = null;
        this.x = x;
        this.y = y;
        this.z = 0;
    }

    public Point(string name, double x, double y, double z)
    {
        this.name = name;
        this.code = null;
        this.x = x;
        this.y = y;
        this.z = z;
    }

    public Point(string name, string code, double x, double y, double z)
    {
        this.name = name;
        this.code = code;
        this.x = x;
        this.y = y;
        this.z = z;
    }
}
\end{lstlisting}

注意以上构造函数中的 this 关键字不可省略, this.name 限定了name 是类中的name,
而 ``='' 后的 name 根据就近原则指的是函数参数中的name。
其含义为将参数name的值赋值给类中的name字段。

有了上面重载形式的构造函数,我们就可以多种方式生成Point类的实例了,如下代码所示:

\begin{lstlisting}
//在Program.cs文件中Main()函数中使用
Point p1 = new Point();
Point p2 = new Point(100.0, 100.0);
Point p3 = new Point(300.0, 300.0, 430.0);
Point p4 = new Point("p4",400.0, 400.0);
Point p5 = new Point("p5",500.0, 500.0, 450.0);
Point p6 = new Point("p6","001", 600.0, 600.0, 460.0);
\end{lstlisting}

构造函数只能用于new操作符生成类的实例,不能显式调用。

构造函数的访问权限一般定义为public,在某些特殊的场景中也可以定义为private。
如果定义为private,该类的new操作只能在该类的静态函数中使用,在其它类中是无法生成该类的实例。

构造函数不能被继承,但子类在其构造函数中可以使用 base 关键字进行调用。

\subsection{类的属性 (property) }

在前面的Point类中我们将各个字段定义为public访问权限,这会导致在任何能引用到类的实例的地方都能
修改类中字段的数据,很显然这不符合封装的原则。在很多情况下,我们需要对类中数据的访问与修改添加
一定的规则,比如测量行业中的坐标 (x、y) 一般情况下不能为负值。

因此,我们需将Point类中的字段访问权限定义为 private 或 protected,代码如下:

\begin{lstlisting}
//在Point.cs文件中定义
public class Point
{
    private string name;
    private string code;
    private double x;
    private double y;
    private double z;

    //此处省略掉Point的构造函数,代码如前面所示
}
\end{lstlisting}

为了支持对Point类中字段的快捷访问,\cs 中引入了\textbf{属性 (Property)}。
属性(Property) 是字段的自然扩展,是 \cs 核心语法,也是后面界面编写的核心。

属性和字段都是类的命名成员,都具有相关的类型,访问字段和属性的语法也相同。

与字段不同的是属性不存储数据。属性本质上讲是两个函数/方法(get与set),也叫做访问器 (accessor),
get代表读数据,set代表写数据,访问器指定在它们的值被读取或写入时需执行的语句。

属性的声明与字段类似,不同的是属性声明以位于定界符 \{ 和 \} 之间的一个 get 访问器和/或一个 set 访问器结束,
不是以分号结束。

Point类属性定义代码如下所示:

\begin{lstlisting}
//在Point.cs文件中定义
public class Point : System.Object
{
    private string name;
    public string Name
    {
        get{ return name;}
        set{ name = value;}
    }

    private string code;
    public string Code
    {
        get{ return code;}
        set{ code = value;}
    }

    private double x;
    public double X
    {
        get{ return x;}
        set{ x = value;}
    }

    private double y;
    public double Y
    {
        get{ return y;}
        set{ y = value;}
    }

    private double z;
    public double Z
    {
        get{ return z;}
        set{ z = value;}
    }

    //此处省略掉Point的构造函数,代码如前面所示    
}
\end{lstlisting}

同时具有 get 访问器和 set 访问器的属性是读写属性 (read-write property),
只有 get 访问器的属性是只读属性 (read-only property),
只有 set 访问器的属性是只写属性 (write-only property)。

get 访问器相当于一个具有属性类型返回值的无形参方法。
除了作为赋值的目标,当在表达式中引用属性时,将调用该属性的 get 访问器以计算该属性的值。

set 访问器相当于具有一个名为 value 的参数并且没有返回类型的方法。
当某个属性作为赋值的目标被引用,或者作为 $++$ 或 $--$ 的操作数被引用时,
将调用 set 访问器,并传入提供新值的实参。

如果属性比较简单,可以使用lambda表达式进行简写:
\begin{lstlisting}
private double x;
public double X
{
    get => x;
    set => x = value;
}
\end{lstlisting}

如果属性只有get部分,如 X 属性可以写成:
\begin{lstlisting}
private double x;
public double X => x;
\end{lstlisting}

如果属性不使用字段的话,可以简写成 get/set 形式,如 X 属性可以写成:
\begin{lstlisting}
//private double x;
public double X
{
    get;
    set;
}
\end{lstlisting}

属性的访问权限一般都定义为public,用于对外交换数据。


\subsection{类的实例方法(method)}

方法也称为函数,是类的一种行为成员,用于实现类的计算或操作。

\begin{itemize}
\item 方法的签名 (signature)

方法的签名在声明该方法的类中必须唯一。方法的签名由方法的名称、
参数类型、参数个数、返回类型、修饰符组成。

方法可以有重载形式,但必须符合重载规则。

方法的参数 (parameter) 表示传递给该方法的值或变量,参数列表可以为空;

方法还有一个返回类型 (return type),用于指定该方法的返回值类型。
如果方法没有返回值,返回类型必须定义为 void,不能不定义或省略。

\item 方法/函数的参数

参数用于向方法传数据,可以传值或传变量。
有四类参数:值参数、引用参数、输出参数和参数数组。

    \begin{itemize}
	\item 值参数 (value parameter)

    值参数用于传递值类型数据。一个值参数相当于一个局部变量,它的初始值来自于该形参传递的实参数据。
    值参数在传递时,实际上是传递值参数的一个拷贝,因此在函数内部对值参数的任何修改不会影响到传值的实参的。
    值参数在传值时可以直接传递数值。

	\item 引用参数 (reference parameter)

    引用参数可用于传递输入和输出的数据。
    在方法执行期间,引用参数与实参变量表示同一存储位置,因此为引用参数传递的实参必须是变量,
    而且必须是具有初始值的变量。

    引用参数使用 ref 修饰符声明。

    下面的交换两个变量值的代码演示了 ref 参数的用法:

\begin{lstlisting}
using System;
class Test
{
    static void Swap(ref int x, ref int y)//交换两个变量的值
     {
        int temp = x;
        x = y;
        y = temp;
    }

    static void Main()
    {
        int i = 1, j = 2; //变量i与j必须初始化
        Swap(ref i, ref j);
        Console.WriteLine("{0} {1}", i, j);  // 输出为 2 1
    }
}
\end{lstlisting}


	\item 输出参数(output parameter)

    输出参数用于函数输出计算数据值。对于输出参数来说,调用方提供的实参的初始值并不重要。
    除此之外,输出参数与引用参数类似。输出参数用 out 修饰符声明的。

    下面的代码演示 out 参数的用法:

\begin{lstlisting}
using System;
class Test
{
    static void Divide(int x, int y, out int result, out int remainder) 
    {
        result = x / y;
        remainder = x % y;
    }

    static void Main() 
    {
        Divide(10, 3, out int res, out int rem);
        Console.WriteLine("{0} {1}", res, rem); // Outputs "3 1"
    }
}
\end{lstlisting}

    第12行代码为较新式的 \cs 语法。

	\item 参数数组 (parameter array)

    参数数组允许向方法传递可变数量的实参。

    参数数组使用 params 修饰符声明。只有方法的最后一个参数才可以是参数数组,
    并且参数数组的类型必须是一维数组类型。

    System.Console 类的 Write 和 WriteLine 方法就是参数数组用法的很好示例,
    它们的声明如下:

\begin{lstlisting}
public class Console
{
    public static void Write(string fmt, params object[] args) {...}
    public static void WriteLine(string fmt, params object[] args) {...}
}
\end{lstlisting}

    在使用参数数组的方法中,参数数组的行为完全就像常规的数组类型参数。
    但是,在具有参数数组的方法的调用中,既可以传递参数数组类型的单个实参,
    也可以传递参数数组的元素类型的任意数目的实参。在后一种情况下,
    将自动创建一个数组实例,并使用给定的实参对它进行初始化。

    传递参数数组类型的单个实参代码示例如下:

\begin{lstlisting}
Console.WriteLine("x={0} y={1} z={2}", x, y, z);
\end{lstlisting}

    上面代码等价于以下语句:

\begin{lstlisting}
string s = "x={0} y={1} z={2}";
object[] args = new object[3];
args[0] = x;
args[1] = y;
args[2] = z;
Console.WriteLine(s, args);
\end{lstlisting}

    \end{itemize}

\end{itemize}

现在我们为Point类增加一个方法:计算两点的平面距离,定义如下:

\begin{lstlisting}
//在Point.cs文件中定义
using System;
public class Point : System.Object
{ 
    //其它省略,代码如前面所示
    public double Distance(Point other)
    {
        double dx = this.x - other.x;
        double dy = this.y - other.y;
        return Math.Sqrt(dx*dx + dy*dy);
    }    
}
\end{lstlisting}

注意这个函数的定义,初学者可能认为这个函数需要两个参数,因为两个点嘛,
实际上这是一个实例类,其本身就是一个点,程序中用this关键字表示的,
因此只需参数传入另一个点Other。

方法的访问权限与类字段相同。



\subsection{类的静态成员}

前边所讲的字段、属性、方法均为类的实例成员,它们为类的每个实例所有,
在类中可以用关键字this进行应用。

如果一个类的成员不属于某个类实例,则应定义为类的静态成员,静态成员属于定义它的类,
通过类名进行引用。

类的静态成员有静态字段,用于存储属于整个类的数据;
有静态属性,用于读写类的静态字段;也有静态方法,用于操作类的静态数据;
类中的常量也属于类的静态字段。

使用 static 修饰符声明的字段、属性、方法为类的静态成员。

试想我们要对前面我们定义的点Point类生成的实例对象进行计数,这个计数的字段最佳的定义方式就是定义为
类的静态字段(实例对象的个数不应该存在于某个实例中,应属于整个点Point类)。

为了能够访问这个计数数字,我们也可为它定义一个静态只读属性,代码如下:

\begin{lstlisting}
//在Point.cs文件中定义
public class Point
{
    private static int count = 0;
    public static int Count => count;

    //此处省略掉Point的其它部分,代码如前面所示    
}
\end{lstlisting}

现在我们要站在上帝的视角来实现两个点的距离计算,同样可以将函数定义为静态的,
由于静态函数中没有实例对象了,所以参数也应该定义为两个点,实现的代码如下:

\begin{lstlisting}
//在Point.cs文件中定义
using System;
public class Point
{ 
    //其它省略,代码如前面所示
    public static double Distance(Point from, Point to)
    {
        double dx = from.x - to.x;
        double dy = from.y - to.y;
        return Math.Sqrt(dx*dx + dy*dy);
    }    
}
\end{lstlisting}

静态方法中不能使用 this 关键字。


\subsection{类的实例成员与静态成员的关系}

一个类的静态方法或属性不能直接通过this关键字调用类的实例成员(如字段、属性、方法等),
但可以通过类的实例对象调用类的实例成员。如上面静态计算两点距离的函数更好的实现方法为:

\begin{lstlisting}
//在Point.cs文件中定义
using System;
public class Point
{ 
    //其它省略,代码如前面所示
    public static double Distance(Point from, Point to)
    {      
        return from.Distance(to);
    }
}
\end{lstlisting}

也就是通过类Point的实例对象from调用实例方法Distance实现计算,这样可以避免写相同或相似的
计算平距的代码两份。

由于静态成员是属于整个类的,因此实例方法或属性是可以直接读写类的静态字段的,
也是可以直接调用类的静态方法和属性的。

比如我们要在类Point添加一个Id属性,实现Id的自增,外部不能修改这个Id值,
则可以在类Point的构造函数中调用Point的静态字段count实现自增并赋值给id实例字段,
这样每个Point类实例都有自己的Id属性值了。完整的Point类代码如下:

\begin{lstlisting}
//在Point.cs文件中定义
public class Point
{
    private static int count = 0;
    public static int Count => count;

    private int id;
    public int Id => id;

    private string name;
    public string Name
    {
        get => name;}
        set => name = value;
    }

    private string code;
    public string Code
    {
        get => code;
        set => code = value;
    }

    private double x;
    public double X
    {
        get => x;
        set => x = value;
    }

    private double y;
    public double Y
    {
        get => y;
        set => y = value;
    }

    private double z;
    public double Z
    {
        get => z;
        set => z = value;
    }

    public Point() //默认构造函数
    {
        this.name = null;
        this.code = null;
        this.x = 0;
        this.y = 0;
        this.z = 0;

        id = ++count;
    }

    public Point(double x, double y)
    {
        this.name = null;
        this.code = null;
        this.x = x;
        this.y = y;
        this.z = 0;

        id = ++count;
    }

    public Point(double x, double y, double z)
    {
        this.name = null;
        this.code = null;
        this.x = x;
        this.y = y;
        this.z = z;

        id = ++count;
    }

    public Point(string name, double x, double y)
    {
        this.name = name;
        this.code = null;
        this.x = x;
        this.y = y;
        this.z = 0;

        id = ++count;
    }

    public Point(string name, double x, double y, double z)
    {
        this.name = name;
        this.code = null;
        this.x = x;
        this.y = y;
        this.z = z;

        id = ++count;
    }

    public Point(string name, string code, double x, double y, double z)
    {
        this.name = name;
        this.code = code;
        this.x = x;
        this.y = y;
        this.z = z;

        id = ++count;
    }

    public override string ToString()
    {
        return $"Id={id}, Name={name}, Code={code}, X={x}, Y={y}, Z={z}";
    }

    public double Distance(Point other)
    {
        double dx = this.x - other.x;
        double dy = this.y - other.y; 
        return Math.Sqrt(dx*dx +dy*dy);
    }

    public static double Distance(Point p1,Point p2)
    {            
        return p1.Distance(p2);
    }
}
\end{lstlisting}

以上代码的第7、8行定义了只读的Id属性,第53、64、75、86、97、108行
为在类Point的所有构造函数中先将计数count加1,赋值给id字段,
从而实现Id属性的自我增长。

第111至114行重写了ToString()函数,这样就可以方便的输出Point的信息了,
在Main函数中的测试代码如下:

\begin{lstlisting}
//在Program.cs文件中Main()函数中使用
Point p1 = new Point();
Console.WriteLine(p1);

Point p2 = new Point(100.0, 100.0);
Console.WriteLine(p2);

Point p3 = new Point(300.0, 300.0, 430.0);
Console.WriteLine(p3);

Point p4 = new Point("p4", 400.0, 400.0);
Console.WriteLine(p4);

Point p5 = new Point("p5", 500.0, 500.0, 450.0);
Console.WriteLine(p5);

Point p6 = new Point("p6", "001", 600.0, 600.0, 460.0);
Console.WriteLine(p6);

Console.WriteLine($"已生成的总点数为:{Point.Count}"); 
Console.WriteLine($"p1--p2点的距离为:{p1.Distance(p2)}"); 
Console.WriteLine($"p3--p4点的距离为:{Point.Distance(p3, p4)}"); 
\end{lstlisting}

测试代码的输出为:
\begin{verbatim}
Id=1, Name=, Code=, X=0, Y=0, Z=0
Id=2, Name=, Code=, X=100, Y=100, Z=0
Id=3, Name=, Code=, X=300, Y=300, Z=430
Id=4, Name=p4, Code=, X=400, Y=400, Z=0
Id=5, Name=p5, Code=, X=500, Y=500, Z=450
Id=6, Name=p6, Code=001, X=600, Y=600, Z=460
已生成的总点数为:6
p1--p2点的距离为:141.4213562373095
p3--p4点的距离为:141.4213562373095
\end{verbatim}


\subsection{类成员扩展阅读}
\cs 中类的成员不仅有字段、方法、属性、常量等,还有其它的一些类型成员,
在本课程中不常用到,所以在此就不详讲了,如果在后续用到了我们再进行讲解。
有些区的同学可以参照下表自行阅读:

\cs 中类的成员如下表所示:
\begin{tabular}{|l|l|}
\hline
成员     &  说明   \\
\hline
常量     &         与类关联的常量值 \\
字段     &         类的变量 \\
方法     &         类可执行的计算和操作 \\
属性     &         与读写类的命名属性相关联的操作 \\
索引器   &      与以数组方式索引类的实例相关联的操作 \\
事件     &         可由类生成的通知 \\
运算符   &      类所支持的转换和表达式运算符 \\
构造函数 &   初始化类的实例或类本身所需的操作 \\
析构函数 &   在永久释放类的实例之前执行的操作 \\
类型     &         类所声明的嵌套类型 \\
\hline
\end{tabular}

类中成员的类型也可以用泛型类型进行定义,则这个类就是泛型类了。

如在下面的代码中,类Pair中就有两个类型参数 T1 和 T2:

\begin{lstlisting}
public class Pair<T1,T2>
{
    public T1 First;
    public T2 Second;
}
\end{lstlisting}

泛型类在实际使用时必须类型化,如下所示:

\begin{lstlisting}
Pair<int,string> pair = new Pair<int,string> { First = 1, Second = “two” };
int i = pair.First;     // TFirst is int
string s = pair.Second; // TSecond is string
\end{lstlisting}

有关泛型的深入阅读请阅读相应的资料。


\section{静态类}

\subsection{静态类}

如果一个类的所有数据成员、属性、方法均与类的单个实例没有关系,则这个类应定义为静态类。
静态类中的所有成员均为静态成员。

类的静态成员通过类名进行引用。

静的态方法和静态属性的定义是在访问修饰符后加static关键字,其它的与实例类中的定义相同。
如我们在前面将排序算法时定义的Sort类,更应该以静态类的形式定义为如下形式:
\begin{lstlisting}
//Sort.cs文件中的内容
using System.Collections.Generic;

namespace ZXY
{
    public static class Sort
    {
        public static int BubbleSort(List<int> arr)
        ...... //此处省略的代码与前面的相同

        public static int QuickSort(List<int> arr)
        ...... //此处省略的代码与前面的相同
    }
}
\end{lstlisting}

在调用时就不用再生成类Sort的示例了,直接用类名Sort进行引用调用,
代码如下:

\begin{lstlisting}
//Program.cs文件中Main函数中的内容

//Sort sort = new Sort();//定义为静态的排序算法,此语句就没有必要
Stopwatch st = new Stopwatch();
st.Start();
//Sort.BubbleSort(list); //冒泡排序,通过类名引用
Sort.QuickSort(list);    //快速排序,通过类名引用
st.Stop();
Console.WriteLine($"运行时间:{st.Elapsed}");
\end{lstlisting}



\subsection{静态构造函数}
 \cs 中所有的类均有一个静态构造函数(static constructor) ,
 它是在第一次加载类本身时执行,用于实现类的初始化操作
 (也是在类的所有构造函数执行之前执行)。

 在构造函数声明前加入 static 修饰符,则显示声明了静态构造函数。

 比如我们可以在前面的Point类中加入静态构造函数,如下所示:

\begin{lstlisting}
namespace ZXY
{
    public class Point
    {
        private static int count;
            
        static Point() //静态构造函数
        {
            count = 0;
            Console.WriteLine("这是Point类静态构造函数");
        }

        //其它代码省略
    }
}
\end{lstlisting}

第7至11行为静态函数内容, 第9行用于初始化静态字段count的值,
第10行用于观察静态构造函数执行的时机,从输出结果可以观察到静态构造函数优先于Point类的
其它函数执行。

静态函数没有重载形式,也不能被显示调用。在类被使用到时由系统自动调用。


\section{继承(Inheritance)}

 继承是指这样一种能力:它可以使用现有类的所有功能,并在无需重新编写原来的类的情况下对这些功能进行扩展。继承符合面向对象设计的开闭原则:对修改封闭、对扩展开发。

通过继承创建的新类称为“子类”或“派生类”。

被继承的类称为“基类”(base class)、“父类”或“超类”(super class)。

继承的过程,就是从一般到特殊的过程,也称为泛化。



%%%%%%%%%%%%%%%%%%%%%%%%%%%%%%%%%%%%
\subsection{继承的写法}

在类名后面添加一个冒号和基类的名称来指定一个基类,省略基类的类默认为从类 System.Object 派生。
在下面的示例中,Point3D 的基类是 Point2D,而 Point2D 的基类是 System.Object。

\begin{lstlisting}
public class Point2D
{
	private string name;
	public string Name
	{
		get => name;
		set => name = value;
	}

	protected double x;
	public double y;
	
	public Point(string name, double x, double y)
	{
		this.name = name;
		this.x = x;
		this.y = y;
	}
}

public class Point3D: Point
{
	private double z;
	public Point3D(string name, double x, double y, double z): base(name, x, y)
	{
		this.z = z;
	}
}
\end{lstlisting}


\subsection{子类继承基类的成员}

子类继承了基类的所有成员,继承意味着子类隐式地将它的基类的所有成员当作自已的成员,无论基类中的成员权限是 private,还是 protected 或 public。


\subsection{继承中的构造函数}

但基类的实例构造函数、静态构造函数和析构函数除外。派生类能够在继承基类的基础上添加新的成员,但是它不能移除继承成员的定义。在前面的示例中,Point3D 从 Point 继承了 x 和 y 字段,并且每个 Point3D 实例均包含三个字段:x、y 和 z。

从某个类类型到它的任何基类类型存在隐式的转换。因此,类类型的变量可以引用该类的实例或任何派生类的实例。例如,对于前面给定的类声明,Point 类型的变量既可以引用 Point 也可以引用 Point3D:
\begin{lstlisting}
Point a = new Point(10, 20);
Point b = new Point3D(10, 20, 30);
\end{lstlisting}

\section{多态与虚函数}
在编程语言和类型论中,多态(polymorphism)是指为不同数据类型的实体提供统一的接口,虚函数是实现多态的工具。


\section{接口(interface)}

\subsection{面向接口编程}

\subsection{接口跳转}


\section{封闭类(sealed class)}

\section{小结}