%!TEX root = ../../clcxsj.tex

\chapter{面向对象程序设计原则}

面向对象程序设计的核心是类与接口,设计良好的程序需要遵循类的七个如下设计原则:

\begin{itemize}
	\item 开闭原则(OCP)
	\item 里氏代换原则
	\item 迪米特原则(最少知道原则)
	\item 单一职责原则
	\item 接口分隔原则
	\item 依赖倒置原则
	\item 组合/聚合复用原则
\end{itemize}

七大原则之间并不是相互孤立的,彼此之间是存在着一定关联,一个可以是另一个原则的加强或基础。
违反其中的某一个原则,可能同时也会违反其余的原则。

开闭原则是面向对象的可复用设计的基石。其他设计原则是实现开闭原则的手段和工具。

一般地,可以把这七个原则分成了以下两个部分:
\begin{itemize}
	\item 设计目标
	开闭原则、里氏代换原则、迪米特原则
	\item 设计方法
	单一职责原则、接口分隔原则、依赖倒置原则、组合/聚合复用原则
\end{itemize}


\section{开闭原则(OCP)}

\subsection{思想}
\emph{ 软件实体(模块,类,方法等)应该对扩展开放,对修改关闭。}

开闭原则英文为 The Open-Closed Principle,简称为OCP,是指在进行面向对象设计中,设计类或其他程序单位时,应该遵循:

\begin{itemize}

    \item 对扩展开放(open)

	某模块的功能是可扩展的,则该模块是扩展开放的。软件系统的功能上的可扩展性要求模块是扩展开放的。
    
    \item 对修改关闭(closed) 的设计原则

	某模块被其他模块调用,如果该模块的源代码不允许修改,则该模块修改关闭的。软件系统的功能上的稳定性,持续性要求模块是修改关闭的。

\end{itemize}

开闭原则是判断面向对象设计是否正确的最基本的原理之一。

根据开闭原则,在设计一个软件系统模块(类,方法)的时候,应该可以在不修改原有的模块(修改关闭)的基础上,能扩展其功能(扩展开放)。

\subsection{原因}

\begin{itemize}
\item{稳定性}

开闭原则要求扩展功能不修改原来的代码,这可以让软件系统在变化中保持稳定。

\item{扩展性}

开闭原则要求对扩展开放,通过扩展提供新的或改变原有的功能,让软件系统具有灵活的可扩展性。
遵循开闭原则的系统设计,可以让软件系统可复用,并且易于维护。
\end{itemize}

\subsection{实现方法}

为了满足开闭原则的对修改关闭原则以及扩展开放原则,应该对软件系统中的不变的部分加以抽象,在面向对象的设计中,

可以把这些不变的部分加以抽象成不变的接口,这些不变的接口可以应对未来的扩展;

接口的最小功能设计原则。根据这个原则,原有的接口要么可以应对未来的扩展;不足的部分可以通过定义新的接口来实现;

模块之间的调用通过抽象接口进行,这样即使实现层发生变化,也无需修改调用方的代码。

接口可以被复用,但接口的实现却不一定能被复用。

接口是稳定的,关闭的,但接口的实现是可变的,开放的。

可以通过对接口的不同实现以及类的继承行为等为系统增加新的或改变系统原来的功能,实现软件系统的柔性扩展。

好处:提高系统的可复用性和可维护性。

简单地说,软件系统是否有良好的接口(抽象)设计是判断软件系统是否满足开闭原则的一种重要的判断基准。现在多把开闭原则等同于面向接口的软件设计。

\subsection{代码示例}


\section{里氏替换原则(Liskov Substitution Principle ,LSP)}
所有引用基类的地方必须能透明地使用其派生类的对象。

也就是说,只有满足以下2个条件的OO设计才可被认为是满足了LSP原则:

不应该在代码中出现if/else之类对派生类类型进行判断的条件。

派生类应当可以替换基类并出现在基类能够出现的任何地方,或者说如果我们把代码中使用基类的地方用它的派生类所代替,代码还能正常工作。

以下代码就违反了LSP定义。


里氏替换原则(LSP)是使代码符合开闭原则的一个重要保证。

同时LSP体现了:

类的继承原则:如果一个派生类的对象可能会在基类出现的地方出现运行错误,则该派生类不应该从该基类继承,或者说,应该重新设计它们之间的关系。

动作正确性保证:从另一个侧面上保证了符合LSP设计原则的类的扩展不会给已有的系统引入新的错误。
示例:

里式替换原则为我们是否应该使用继承提供了判断的依据,不再是简单地根据两者之间是否有相同之处来说使用继承。

里式替换原则的引申意义:子类可以扩展父类的功能,但不能改变父类原有的功能。

具体来说:

子类可以实现父类的抽象方法,但不能覆盖父类的非抽象方法。
子类中可以增加自己特有的方法。
当子类的方法重载父类的方法时,方法的前置条件(即方法的输入/入参)要比父类方法的输入参数更宽松。
当子类的方法实现父类的方法时(重载/重写或实现抽象方法)的后置条件(即方法的输出/返回值)要比父类更严格或相等。

\section{迪米特原则(最少知道原则)(Law of Demeter ,LoD)}

迪米特原则(Law of Demeter)又叫最少知道原则(Least Knowledge Principle),可以简单说成:talk only to your immediate friends,只与你直接的朋友们通信,不要跟“陌生人”说话。

对于面向OOD来说,又被解释为下面两种方式:

1)一个软件实体应当尽可能少地与其他实体发生相互作用。

2)每一个软件单位对其他的单位都只有最少的知识,而且局限于那些与本单位密切相关的软件单位。

朋友圈的确定
“朋友”条件:

当前对象本身(this)
以参量形式传入到当前对象方法中的对象
当前对象的实例变量直接引用的对象
当前对象的实例变量如果是一个聚集,那么聚集中的元素也都是朋友
当前对象所创建的对象
任何一个对象,如果满足上面的条件之一,就是当前对象的“朋友”,否则就是“陌生人”。

迪米特原则的优缺点

迪米特原则的初衷在于降低类之间的耦合。由于每个类尽量减少对其他类的依赖,因此,很容易使得系统的功能模块功能独立,相互之间不存在(或很少有)依赖关系。

迪米特原则不希望类直接建立直接的接触。如果真的有需要建立联系,也希望能通过它的友元类来转达。因此,应用迪米特原则有可能造成的一个后果就是:系统中存在大量的中介类,这些类之所以存在完全是为了传递类之间的相互调用关系,这在一定程度上增加了系统的复杂度。

例如,购房者要购买楼盘A、B、C中的楼,他不必直接到楼盘去买楼,而是可以通过一个售楼处去了解情况,这样就减少了购房者与楼盘之间的耦合,如图所示。

\section{单一职责原则}
永远不要让一个类存在多个改变的理由。

换句话说,如果一个类需要改变,改变它的理由永远只有一个。如果存在多个改变它的理由,就需要重新设计该类。

单一职责原则原则的核心含意是:只能让一个类/接口/方法有且仅有一个职责。

为什么一个类不能有多于一个以上的职责?
如果一个类具有一个以上的职责,那么就会有多个不同的原因引起该类变化,而这种变化将影响到该类不同职责的使用者(不同用户):

一方面,如果一个职责使用了外部类库,则使用另外一个职责的用户却也不得不包含这个未被使用的外部类库。
另一方面,某个用户由于某个原因需要修改其中一个职责,另外一个职责的用户也将受到影响,他将不得不重新编译和配置。
这违反了设计的开闭原则,也不是我们所期望的。
职责的划分
既然一个类不能有多个职责,那么怎么划分职责呢?

Robert.C Martin给出了一个著名的定义:所谓一个类的一个职责是指引起该类变化的一个原因。

如果你能想到一个类存在多个使其改变的原因,那么这个类就存在多个职责。

\section{接口分隔原则(Interface Segregation Principle ,ISP)}
不能强迫用户去依赖那些他们不使用的接口。

换句话说,使用多个专门的接口比使用单一的总接口总要好。

它包含了2层意思:

接口的设计原则:接口的设计应该遵循最小接口原则,不要把用户不使用的方法塞进同一个接口里。如果一个接口的方法没有被使用到,则说明该接口过胖,应该将其分割成几个功能专一的接口。

接口的依赖(继承)原则:如果一个接口a继承另一个接口b,则接口a相当于继承了接口b的方法,那么继承了接口b后的接口a也应该遵循上述原则:不应该包含用户不使用的方法。 反之,则说明接口a被b给污染了,应该重新设计它们的关系。

如果用户被迫依赖他们不使用的接口,当接口发生改变时,他们也不得不跟着改变。换而言之,一个用户依赖了未使用但被其他用户使用的接口,当其他用户修改该接口时,依赖该接口的所有用户都将受到影响。这显然违反了开闭原则,也不是我们所期望的。

总而言之,接口分隔原则指导我们:

一个类对一个类的依赖应该建立在最小的接口上

建立单一接口,不要建立庞大臃肿的接口

尽量细化接口,接口中的方法尽量少

接口分隔原则的优点和适度原则
接口分隔原则从对接口的使用上为我们对接口抽象的颗粒度建立了判断基准:在为系统设计接口的时候,使用多个专门的接口代替单一的胖接口。

符合高内聚低耦合的设计思想,从而使得类具有很好的可读性、可扩展性和可维护性。

注意适度原则,接口分隔要适度,避免产生大量的细小接口。

单一职责原则和接口分隔原则的区别
单一职责强调的是接口、类、方法的职责是单一的,强调职责,方法可以多,针对程序中实现的细节;

接口分隔原则主要是约束接口,针对抽象、整体框架。

\section{依赖倒置原则(Dependency Inversion Principle ,DIP)}
A. 高层模块不应该依赖于低层模块,二者都应该依赖于抽象

B. 抽象不应该依赖于细节,细节应该依赖于抽象

C.针对接口编程,不要针对实现编程。

依赖:在程序设计中,如果一个模块a使用/调用了另一个模块b,我们称模块a依赖模块b。

高层模块与低层模块:往往在一个应用程序中,我们有一些低层次的类,这些类实现了一些基本的或初级的操作,我们称之为低层模块;另外有一些高层次的类,这些类封装了某些复杂的逻辑,并且依赖于低层次的类,这些类我们称之为高层模块。

依赖倒置(Dependency Inversion):
面向对象程序设计相对于面向过程(结构化)程序设计而言,依赖关系被倒置了。因为传统的结构化程序设计中,高层模块总是依赖于低层模块。

\section{组合/聚合复用原则(Composite/Aggregate Reuse Principle ,CARP)}
尽量使用组合/聚合,不要使用类继承。

即在一个新的对象里面使用一些已有的对象,使之成为新对象的一部分,新对象通过向这些对象的委派达到复用已有功能的目的。就是说要尽量的使用合成和聚合,而不是继承关系达到复用的目的。

组合和聚合都是关联的特殊种类。

聚合表示整体和部分的关系,表示“拥有”。组合则是一种更强的“拥有”,部分和整体的生命周期一样。

组合的新的对象完全支配其组成部分,包括它们的创建和湮灭等。一个组合关系的成分对象是不能与另一个组合关系共享的。

组合是值的聚合(Aggregation by Value),而一般说的聚合是引用的聚合(Aggregation by Reference)。

在面向对象设计中,有两种基本的办法可以实现复用:第一种是通过组合/聚合,第二种就是通过继承。

什么时候才应该使用继承
只有当以下的条件全部被满足时,才应当使用继承关系:

1)派生类是基类的一个特殊种类,而不是基类的一个角色,也就是区分"Has-A"和"Is-A"。只有"Is-A"关系才符合继承关系,"Has-A"关系应当用聚合来描述。

2)永远不会出现需要将派生类换成另外一个类的派生类的情况。如果不能肯定将来是否会变成另外一个派生类的话,就不要使用继承。

3)派生类具有扩展基类的责任,而不是具有置换掉(override)或注销掉(Nullify)基类的责任。如果一个派生类需要大量的置换掉基类的行为,那么这个类就不应该是这个基类的派生类。

4)只有在分类学角度上有意义时,才可以使用继承。

总的来说:

如果语义上存在着明确的"Is-A"关系,并且这种关系是稳定的、不变的,则考虑使用继承;如果没有"Is-A"关系,或者这种关系是可变的,使用组合。另外一个就是只有两个类满足里氏替换原则的时候,才可能是"Is-A" 关系。也就是说,如果两个类是"Has-A"关系,但是设计成了继承,那么肯定违反里氏替换原则。

错误的使用继承而不是组合/聚合的一个常见原因是错误的把"Has-A"当成了"Is-A" 。"Is-A"代表一个类是另外一个类的一种;"Has-A"代表一个类是另外一个类的一个角色,而不是另外一个类的特殊种类。
