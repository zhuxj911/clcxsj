%!TEX root = ../clcxsj.tex

%%
% 对课程进行介绍
%%

\chapter{教学计划及课程目标}

\section{教学计划}

\subsection{教学课时}
该课程标准学时为64学时。本学期为60学时,上课15次30学时,上机15次30学时。

必修课,考试。笔试(60\%)+上机实验与作业(40\%)

\subsection{学习内容}

学习面向对象的程序设计与编写;

学习基本的测量算法程序编写,学习软件界面的编写。

\section{编程预言的选择}

注意:该课程不是再次学习某种编程语言,而是运用某种编程语言进行
测量数据处理及测量算法的编写。

可能大家只学习了C语言,C语言是面向过程的功能强大的语言,执行效率高,但界面程序编写较复杂。
现代编程语言更多的是面向对象的编程语言,高效而且兼容C语言的是C++,但界面的编写仍较复杂。
因此我们选择更加现代化的编程语言C\#,它语法优美,界面编写方式有 WinForm 与 WPF方式,虽然
执行效率没有C++高,但在Windows7、Windows8/8.1、Windows10中都几乎预装了运行库 .Net Frame,
并且得到了Microsoft的大力推广,Microsoft的许多软件都在使用C\#开发,学习与开发成本相比与C++显著减少,
效率显著提升。

C\#的语法与JAVA的许多语法是极其相似的,现代软件工程的许多方法也可以用C\#去实现。因此,在本课程中我们将
学习C\#编程语言,并且会学习如何运用C\#语言进行测量程序设计及测量数据处理。

现代软件基本上都具有简洁易用的界面,我们还将学习WPF的界面编写技术,为我们的程序设计处美观简洁的界面,
这些都包含在C\#之中。

\section{编程环境与工具}

本课程的基本编程工具为Visual Studio,版本理论上为2010及更新版,
推荐大家用Visual Studio 2015或2017版(目前的最新版)。

\section{参考教材}

C\# 语言及WPF界面编写参考马骏主编,人民邮电出版社出版,《C\#程序设计及应用教程》第3版。

测量算法参考本教程(一直更新中......)。