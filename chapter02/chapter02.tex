%!TEX root = ../clcxsj.tex

\chapter{C\#语言基础}

由于我们都学习过C语言,在此我们主要讲解C\#中不同于C的一些基本语法。

\section{ C\#程序的组织结构}

 从以上C\#示例代码可以看出,
 C\# 中的组织结构的关键概念是程序 (program)、命名空间 (namespace)、类型 (type)、成员 (member) 和程序集 (assembly)。
 C\# 程序由一个或多个源文件组成。程序中声明类型,类型包含成员,并且可按命名空间进行组织。类和接口就是类型的实例。
 字段 (field)、方法、属性和事件是成员的实例。在编译 C\# 程序时,它们被物理地打包为程序集。
 程序集通常具有文件扩展名 .exe 或 .dll,具体取决于它们是实现应用程序 (application) 还是实现库 (library)。

\section{C\#中的数据类型}
C\#中的数据类型可分为两类:值类型 (value type) 和引用类型 (reference type)。
值类型的变量直接包含它们的数据,而引用类型的变量存储对它们的数据的引用,后者称为对象。

\subsection{值类型}
C\# 的值类型进一步划分为简单类型 (simple type)、枚举类型 (enum type)、结构类型 (struct type)
和可以为 null 的类型 (nullable type)。
对于值类型,每个变量都有它们自己的数据副本(除 ref 和 out 参数变量外),
因此对一个变量的操作不可能影响另一个变量。

\begin{tabular}{|l|l|}
\hline
类别   & 说明 \\
\hline
简单类型   & 有符号整型:sbyte、short、int、long \\
                    & 无符号整型:byte、ushort、uint、ulong \\
                    & Unicode 字符:char \\
                   &  IEEE 浮点:float、double \\
                    & 高精度小数型:decimal \\
                    & 布尔:bool \\
\hline
结构类型 & struct S {...} 形式的用户定义的类型 \\
枚举类型  & enum E {...} 形式的用户定义的类型 \\
可以为 null 的类型 & 其他所有具有 null 值的值类型的扩展 \\
\hline
\end{tabular}


\subsection{引用类型}
C\# 的引用类型进一步划分为类类型 (class type)、接口类型 (interface type)、
数组类型 (array type) 和委托类型 (delegate type)。
对于引用类型,两个变量可能引用同一个对象,因此对一个变量的操作可能影响另一个变量所引用的对象。

\begin{tabular}{|l|l|}
\hline
类别   & 说明 \\
\hline
类型  &  所有其他类型的最终基类:object \\
              &  Unicode 字符串:string \\
              &  class C {...} 形式的用户定义的类型 \\
\hline
接口类型  & interface I {...} 形式的用户定义的类型  \\
数组类型  & 一维和多维数组,例如 int[] 和 int[,]  \\
委托类型  & delegate int D(...) 形式的用户定义的类型  \\
\hline
\end{tabular}

C\#中的 string不是值类型数据,而是引用类型数据。

\subsection{自定义类型}
C\# 程序使用类型声明 (type declaration) 创建新类型。类型声明指定新类型的名称和成员。
在 C\# 类型分类中,有五类是用户可定义的:类类型 (class type)、结构类型 (struct type)、
接口类型 (interface type)、枚举类型 (enum type) 和委托类型 (delegate type)。


类类型定义了一个包含数据成员(字段)和函数成员(方法、属性等)的数据结构。
类类型支持单一继承和多态,这些是派生类可用来扩展和专用化基类的机制。

结构类型与类类型相似,表示一个带有数据成员和函数成员的结构。
但是,与类不同,结构是一种值类型,并且不需要堆分配。
结构类型不支持用户指定的继承,并且所有结构类型都隐式地从类型 object 继承。

接口类型定义了一个协定,作为一个公共函数成员的命名集。
实现某个接口的类或结构必须提供该接口的函数成员的实现。
一个接口可以从多个基接口继承,而一个类或结构可以实现多个接口。

委托类型表示对具有特定参数列表和返回类型的方法的引用。
通过委托,我们能够将方法作为实体赋值给变量和作为参数传递。
委托类似于在其他某些语言中的函数指针的概念,但是与函数指针不同,委托是面向对象的,并且是类型安全的。

类类型、结构类型、接口类型和委托类型都支持泛型,因此可以通过其他类型将其参数化。

枚举类型是具有命名常量的独特的类型。每种枚举类型都具有一个基础类型,
该基础类型必须是八种整型之一。枚举类型的值集和它的基础类型的值集相同。

C\# 支持由任何类型组成的一维和多维数组。与以上列出的类型不同,数组类型不必声明就可以使用。
实际上,数组类型是通过在某个类型名后加一对方括号来构造的。
例如,int[] 是一维 int 数组,int[,] 是二维 int 数组,int[][] 是一维 int 数组的一维数组。

可以为 null 的类型也不必声明就可以使用。对于每个不可以为 null 的值类型 T,
都有一个相应的可以为 null 的类型 T?,该类型可以容纳附加值 null。
例如,int? 类型可以容纳任何 32 位整数或 null 值。

\subsection{装箱与拆箱}
C\# 的类型系统是统一的,因此任何类型的值都可以按对象处理。C\# 中的每个类型直接或间接地从 object 类类型派生,
而 object 是所有类型的最终基类。引用类型的值都被视为 object 类型,被简单地当作对象来处理。
值类型的值则通过对其执行装箱和拆箱操作按对象处理。下面的示例将 int 值转换为 object,然后又转换回 int。

\begin{lstlisting}[language=C]
using System;
class Test
{
    static void Main() {
        int i = 123;
        object o = i;      // Boxing
        int j = (int)o;     // Unboxing
    }
}
\end{lstlisting}

当将值类型的值转换为类型 object 时,将分配一个对象实例(也称为“箱子”)以包含该值,
并将值复制到该箱子中。反过来,当将一个 object 引用强制转换为值类型时,
将检查所引用的对象是否含有正确的值类型,如果有,则将箱子中的值复制出来。

\subsection{表达式}
表达式由操作数 (operand) 和运算符 (operator) 构成。表达式的运算符指示对操作数适用什么样的运算。
运算符的示例包括+、-、*、/ 和 new。操作数的示例包括文本、字段、局部变量和表达式。

\begin{tabular}{|l|l|}
\hline
类别            & 说明 \\
\hline
x.m            &          成员访问  \\
x(...)         &           方法和委托调用  \\
x[...]         &           数组和索引器访问  \\
x++            &        后增量  \\
x--            &           后减量  \\
new T(...)     &       对象和委托创建  \\
new T(...){...}&      使用初始值设定项创建对象  \\
new {...}      &         匿名对象初始值设定项  \\
new T[...]     &        数组创建  \\
typeof(T)      &      获取 T 的 System.Type 对象  \\
checked(x)     &    在 checked 上下文中计算表达式  \\
unchecked(x)   & 在 unchecked 上下文中计算表达式  \\
default(T)     &       获取类型 T 的默认值  \\
delegate {...} &      匿名函数(匿名方法)  \\
(T)x           &             将 x 显式转换为类型 T  \\
await x        &         异步等待 x 完成  \\
x is T         &            如果 x 为 T,则返回 true,否则返回 false  \\
x as T         &           返回转换为类型 T 的 x,如果 x 不是 T 则返回 null  \\
(T x) => y     &       匿名函数(lambda 表达式)  \\
\hline
\end{tabular}

其它的与C或C++语言基本相同。

\subsection{ 语句 }
程序的操作是使用语句 (statement) 来表示的。

声明语句 (declaration statement) 用于声明局部变量和常量。

表达式语句 (expression statement) 用于对表达式求值。
可用作语句的表达式包括方法调用、使用 new 运算符的对象分配、
使用 = 和复合赋值运算符的赋值、使用 ++ 和 -- 运算符的增量和减量运算以及 await 表达式。

选择语句 (selection statement) 用于根据表达式的值从若干个给定的语句中选择一个来执行。这一组中有 if 和 switch 语句。

迭代语句 (iteration statement) 用于重复执行嵌入语句。这一组中有 while、do、for 和 foreach 语句。

foreach语句示例代码如下:
\begin{lstlisting}[language=C]
static void Main(string[] args) {
    foreach (string s in args) {
        Console.WriteLine(s);
    }
}
\end{lstlisting}


跳转语句 (jump statement) 用于转移控制。这一组中有 break、continue、goto、throw、return 和 yield 语句。
yield语句
\begin{lstlisting}[language=C]
static IEnumerable<int> Range(int from, int to) {
    for (int i = from; i < to; i++) {
        yield return i;
    }
    yield break;
}
static void Main() {
    foreach (int x in Range(-10,10)) {
        Console.WriteLine(x);
    }
}
\end{lstlisting}

try...catch 语句用于捕获在块的执行期间发生的异常,try...finally 语句用于指定终止代码,不管是否发生异常,该代码都始终要执行。
throw 和 try语句
 \begin{lstlisting}[language=C]
 static double Divide(double x, double y) {
    if (y == 0) throw new DivideByZeroException();
    return x / y;
}
static void Main(string[] args) {
    try {
        if (args.Length != 2) {
            throw new Exception("Two numbers required");
        }
        double x = double.Parse(args[0]);
        double y = double.Parse(args[1]);
        Console.WriteLine(Divide(x, y));
    }
    catch (Exception e) {
        Console.WriteLine(e.Message);
    }
    finally {
        Console.WriteLine(“Good bye!”);
    }
}
\end{lstlisting}

checked 语句和 unchecked 语句用于控制整型算术运算和转换的溢出检查上下文。
 \begin{lstlisting}[language=C]
static void Main() {
    int i = int.MaxValue;
    checked {
        Console.WriteLine(i + 1);       // Exception
    }
    unchecked {
        Console.WriteLine(i + 1);       // Overflow
    }
}
\end{lstlisting}

lock 语句用于获取某个给定对象的互斥锁,执行一个语句,然后释放该锁。
 \begin{lstlisting}[language=C]
class Account
{
    decimal balance;
    public void Withdraw(decimal amount) {
        lock (this) {
            if (amount > balance) {
                throw new Exception("Insufficient funds");
            }
            balance -= amount;
        }
    }
}
\end{lstlisting}

using 语句用于获得一个资源,执行一个语句,然后释放该资源。
 \begin{lstlisting}[language=C]
static void Main() {
    using (TextWriter w = File.CreateText("test.txt")) {
        w.WriteLine("Line one");
        w.WriteLine("Line two");
        w.WriteLine("Line three");
    }
}
\end{lstlisting}


