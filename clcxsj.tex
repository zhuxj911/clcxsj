%-*- coding: UTF-8 -*-
%clcxsj.tex
%测量程序设计

\documentclass[UTF8, A4]{ctexbook}

 \usepackage{graphicx}
 \usepackage{amsmath}
 \usepackage{mathtools}
 \usepackage{fancyhdr}
 \usepackage{listings} %%专用于代码排版的LaTeX宏包, 可对关键字、注释、字符串等
                                         %%使用不同的字体和颜色,也可为代码添加边框、北京背景等风格
 \usepackage{xcolor}  %%对代码进行色彩支持
    
 \pagestyle{fancy}
 \fancyhf{} % delete current setting for header and footer
 \fancyhead[LE,RO]{\bfseries\thepage}
 \fancyhead[LO]{\bfseries\rightmark}
 \fancyhead[RE]{\bfseries\leftmark}

 % \renewcommand{\headrulewidth}{0.5pt}
 % \renewcommand{\footrulewidth}{0pt}
  \addtolength{\headheight}{13pt} % make space for the rule
 % \fancypagestyle{plain}{ \fancyhead{}  \renewcommand{\headrulewidth}{0pt} }

 \CTEXsetup[number={\arabic{chapter}}]{chapter}
 \renewcommand{\chaptermark}[1]{\markboth{#1}{}}
 \renewcommand{\sectionmark}[1]{\markright{\thesection\ #1}}
 
 \title{\heiti \zihao{2} 测量程序设计(C\#--WPF) }
 \author{\kaishu \zihao{-3}朱学军 }
 \date{ 2018年3月 }

\begin{document}

\lstset{basicstyle=\small, 
        numbers=left, %设置行号位置
        numberstyle=\small, %设置行号大小
        keywordstyle=\color{blue}, %设置关键字颜色
        commentstyle=\color[cmyk]{1,0,1,0}, %设置注释颜色
        frame=shadowbox, rulesepcolor=\color{red!20!green!20!blue!20}, %设置边框格式 shadowbox
        breaklines, %自动折行
        extendedchars=false, %解决代码跨页时,章节标题,页眉等汉字不显示的问题
        xleftmargin=2em,xrightmargin=2em, aboveskip=1em, %设置边距
        tabsize=4, %设置tab空格数
        showspaces=false %不显示空格
        %%escapeinside=``{\%}{)} 
       }%%设置 lstlisting 环境所用的公共参数
\maketitle

\tableofcontents

%%%%%%%%%%%%%%%%%%%%%%%%%%%%%%
%% 正文部分
%%%%%%%%%%%%%%%%%%%%%%%%%%%%%%
%!TEX root = ../clcxsj.tex

\chapter{从C走向C\#}

\section{C/S程序设计模式}
改变C语言中将代码写入main函数中的习惯;

改变C\#语言中将代码写入Main函数中的习惯;

改变在WinForm中直接写入代码的习惯;

以上这些习惯将带来一系列的问题,在团队开发与多人协作尤其如此,
如:不能进行unittest、git代码合并时会引发大量的冲突。

\section{从面向过程走向面向对象的程序设计 }

良好的面向过程设计程序设计程序是可以很好的转向面向对象的程序设计的。

 
\begin{lstlisting}[language=C]
#include <math.h>

typedef struct _point {
    char name;
    double x, y, z;
}Point;

typedef struct _circle {
    Point center;
    double r;
    double area;
    double length;
}Circle;

//计算两点的距离
double distance(Point * p1, Point * p2){
    double dx = p2->x - p1->x;
    double dy = p2->y - p1->y;
    return sqrt(dx*dx + dy*dy);
}

//判断两圆是否相交
bool isIntersect(Circle * c1, Circle * c2){
    double d = distance(&c1->center, &c2->center);
    if (c1->r + c2->r > d)
        return false;
    else
        return true;
}

\end{lstlisting}

% \lstinputlisting{source_filename.py}
% \lstinputlisting[language=Python]{source_filename.py}
% \lstinputlisting[language=Python, firstline=37, lastline=45]{source_filename.py}

\section{ 结构体(struct) }

C语言中有 struct,在C++中对 struct 进行了扩展,其实质就是一class。

\section{ 类(class) }

class是绝大多数的面向对象程序设计语言的关键词。

程序或软件的基本概念是:程序 = 数据结构 + 算法

从程序设计语言的语法角度分析,class 是 数据与函数的复合体,是符合上述程序设计原则的。
 %%第1章 从C走向C\#
%!TEX root = ../clcxsj.tex

\chapter{C\#语言基础}

由于我们都学习过C语言,在此我们主要讲解C\#中不同于C的一些基本语法。

\section{ C\#程序的组织结构}

 从以上C\#示例代码可以看出,
 C\# 中的组织结构的关键概念是程序 (program)、命名空间 (namespace)、类型 (type)、成员 (member) 和程序集 (assembly)。
 C\# 程序由一个或多个源文件组成。程序中声明类型,类型包含成员,并且可按命名空间进行组织。类和接口就是类型的实例。
 字段 (field)、方法、属性和事件是成员的实例。在编译 C\# 程序时,它们被物理地打包为程序集。
 程序集通常具有文件扩展名 .exe 或 .dll,具体取决于它们是实现应用程序 (application) 还是实现库 (library)。

\section{C\#中的数据类型}
C\#中的数据类型可分为两类:值类型 (value type) 和引用类型 (reference type)。
值类型的变量直接包含它们的数据,而引用类型的变量存储对它们的数据的引用,后者称为对象。

\subsection{值类型}
C\# 的值类型进一步划分为简单类型 (simple type)、枚举类型 (enum type)、结构类型 (struct type)
和可以为 null 的类型 (nullable type)。
对于值类型,每个变量都有它们自己的数据副本(除 ref 和 out 参数变量外),
因此对一个变量的操作不可能影响另一个变量。

\begin{tabular}{|l|l|}
\hline
类别   & 说明 \\
\hline
简单类型   & 有符号整型:sbyte、short、int、long \\
                    & 无符号整型:byte、ushort、uint、ulong \\
                    & Unicode 字符:char \\
                   &  IEEE 浮点:float、double \\
                    & 高精度小数型:decimal \\
                    & 布尔:bool \\
\hline
结构类型 & struct S {...} 形式的用户定义的类型 \\
枚举类型  & enum E {...} 形式的用户定义的类型 \\
可以为 null 的类型 & 其他所有具有 null 值的值类型的扩展 \\
\hline
\end{tabular}


\subsection{引用类型}
C\# 的引用类型进一步划分为类类型 (class type)、接口类型 (interface type)、
数组类型 (array type) 和委托类型 (delegate type)。
对于引用类型,两个变量可能引用同一个对象,因此对一个变量的操作可能影响另一个变量所引用的对象。

\begin{tabular}{|l|l|}
\hline
类别   & 说明 \\
\hline
类型  &  所有其他类型的最终基类:object \\
              &  Unicode 字符串:string \\
              &  class C {...} 形式的用户定义的类型 \\
\hline
接口类型  & interface I {...} 形式的用户定义的类型  \\
数组类型  & 一维和多维数组,例如 int[] 和 int[,]  \\
委托类型  & delegate int D(...) 形式的用户定义的类型  \\
\hline
\end{tabular}

C\#中的 string不是值类型数据,而是引用类型数据。

\subsection{自定义类型}
C\# 程序使用类型声明 (type declaration) 创建新类型。类型声明指定新类型的名称和成员。
在 C\# 类型分类中,有五类是用户可定义的:类类型 (class type)、结构类型 (struct type)、
接口类型 (interface type)、枚举类型 (enum type) 和委托类型 (delegate type)。


类类型定义了一个包含数据成员(字段)和函数成员(方法、属性等)的数据结构。
类类型支持单一继承和多态,这些是派生类可用来扩展和专用化基类的机制。

结构类型与类类型相似,表示一个带有数据成员和函数成员的结构。
但是,与类不同,结构是一种值类型,并且不需要堆分配。
结构类型不支持用户指定的继承,并且所有结构类型都隐式地从类型 object 继承。

接口类型定义了一个协定,作为一个公共函数成员的命名集。
实现某个接口的类或结构必须提供该接口的函数成员的实现。
一个接口可以从多个基接口继承,而一个类或结构可以实现多个接口。

委托类型表示对具有特定参数列表和返回类型的方法的引用。
通过委托,我们能够将方法作为实体赋值给变量和作为参数传递。
委托类似于在其他某些语言中的函数指针的概念,但是与函数指针不同,委托是面向对象的,并且是类型安全的。

类类型、结构类型、接口类型和委托类型都支持泛型,因此可以通过其他类型将其参数化。

枚举类型是具有命名常量的独特的类型。每种枚举类型都具有一个基础类型,
该基础类型必须是八种整型之一。枚举类型的值集和它的基础类型的值集相同。

C\# 支持由任何类型组成的一维和多维数组。与以上列出的类型不同,数组类型不必声明就可以使用。
实际上,数组类型是通过在某个类型名后加一对方括号来构造的。
例如,int[] 是一维 int 数组,int[,] 是二维 int 数组,int[][] 是一维 int 数组的一维数组。

可以为 null 的类型也不必声明就可以使用。对于每个不可以为 null 的值类型 T,
都有一个相应的可以为 null 的类型 T?,该类型可以容纳附加值 null。
例如,int? 类型可以容纳任何 32 位整数或 null 值。

\subsection{装箱与拆箱}
C\# 的类型系统是统一的,因此任何类型的值都可以按对象处理。C\# 中的每个类型直接或间接地从 object 类类型派生,
而 object 是所有类型的最终基类。引用类型的值都被视为 object 类型,被简单地当作对象来处理。
值类型的值则通过对其执行装箱和拆箱操作按对象处理。下面的示例将 int 值转换为 object,然后又转换回 int。

\begin{lstlisting}[language=C]
using System;
class Test
{
    static void Main() {
        int i = 123;
        object o = i;      // Boxing
        int j = (int)o;     // Unboxing
    }
}
\end{lstlisting}

当将值类型的值转换为类型 object 时,将分配一个对象实例(也称为“箱子”)以包含该值,
并将值复制到该箱子中。反过来,当将一个 object 引用强制转换为值类型时,
将检查所引用的对象是否含有正确的值类型,如果有,则将箱子中的值复制出来。

\subsection{表达式}
表达式由操作数 (operand) 和运算符 (operator) 构成。表达式的运算符指示对操作数适用什么样的运算。
运算符的示例包括+、-、*、/ 和 new。操作数的示例包括文本、字段、局部变量和表达式。

\begin{tabular}{|l|l|}
\hline
类别            & 说明 \\
\hline
x.m            &          成员访问  \\
x(...)         &           方法和委托调用  \\
x[...]         &           数组和索引器访问  \\
x++            &        后增量  \\
x--            &           后减量  \\
new T(...)     &       对象和委托创建  \\
new T(...){...}&      使用初始值设定项创建对象  \\
new {...}      &         匿名对象初始值设定项  \\
new T[...]     &        数组创建  \\
typeof(T)      &      获取 T 的 System.Type 对象  \\
checked(x)     &    在 checked 上下文中计算表达式  \\
unchecked(x)   & 在 unchecked 上下文中计算表达式  \\
default(T)     &       获取类型 T 的默认值  \\
delegate {...} &      匿名函数(匿名方法)  \\
(T)x           &             将 x 显式转换为类型 T  \\
await x        &         异步等待 x 完成  \\
x is T         &            如果 x 为 T,则返回 true,否则返回 false  \\
x as T         &           返回转换为类型 T 的 x,如果 x 不是 T 则返回 null  \\
(T x) => y     &       匿名函数(lambda 表达式)  \\
\hline
\end{tabular}

其它的与C或C++语言基本相同。

\subsection{ 语句 }
程序的操作是使用语句 (statement) 来表示的。

声明语句 (declaration statement) 用于声明局部变量和常量。

表达式语句 (expression statement) 用于对表达式求值。
可用作语句的表达式包括方法调用、使用 new 运算符的对象分配、
使用 = 和复合赋值运算符的赋值、使用 ++ 和 -- 运算符的增量和减量运算以及 await 表达式。

选择语句 (selection statement) 用于根据表达式的值从若干个给定的语句中选择一个来执行。这一组中有 if 和 switch 语句。

迭代语句 (iteration statement) 用于重复执行嵌入语句。这一组中有 while、do、for 和 foreach 语句。

foreach语句示例代码如下:
\begin{lstlisting}[language=C]
static void Main(string[] args) {
    foreach (string s in args) {
        Console.WriteLine(s);
    }
}
\end{lstlisting}


跳转语句 (jump statement) 用于转移控制。这一组中有 break、continue、goto、throw、return 和 yield 语句。
yield语句
\begin{lstlisting}[language=C]
static IEnumerable<int> Range(int from, int to) {
    for (int i = from; i < to; i++) {
        yield return i;
    }
    yield break;
}
static void Main() {
    foreach (int x in Range(-10,10)) {
        Console.WriteLine(x);
    }
}
\end{lstlisting}

try...catch 语句用于捕获在块的执行期间发生的异常,try...finally 语句用于指定终止代码,不管是否发生异常,该代码都始终要执行。
throw 和 try语句
 \begin{lstlisting}[language=C]
 static double Divide(double x, double y) {
    if (y == 0) throw new DivideByZeroException();
    return x / y;
}
static void Main(string[] args) {
    try {
        if (args.Length != 2) {
            throw new Exception("Two numbers required");
        }
        double x = double.Parse(args[0]);
        double y = double.Parse(args[1]);
        Console.WriteLine(Divide(x, y));
    }
    catch (Exception e) {
        Console.WriteLine(e.Message);
    }
    finally {
        Console.WriteLine(“Good bye!”);
    }
}
\end{lstlisting}

checked 语句和 unchecked 语句用于控制整型算术运算和转换的溢出检查上下文。
 \begin{lstlisting}[language=C]
static void Main() {
    int i = int.MaxValue;
    checked {
        Console.WriteLine(i + 1);       // Exception
    }
    unchecked {
        Console.WriteLine(i + 1);       // Overflow
    }
}
\end{lstlisting}

lock 语句用于获取某个给定对象的互斥锁,执行一个语句,然后释放该锁。
 \begin{lstlisting}[language=C]
class Account
{
    decimal balance;
    public void Withdraw(decimal amount) {
        lock (this) {
            if (amount > balance) {
                throw new Exception("Insufficient funds");
            }
            balance -= amount;
        }
    }
}
\end{lstlisting}

using 语句用于获得一个资源,执行一个语句,然后释放该资源。
 \begin{lstlisting}[language=C]
static void Main() {
    using (TextWriter w = File.CreateText("test.txt")) {
        w.WriteLine("Line one");
        w.WriteLine("Line two");
        w.WriteLine("Line three");
    }
}
\end{lstlisting}


 %%第2章 C\#语言基础与Visual Studio编程环境
%!TEX root = ../clcxsj.tex

\chapter{C\#类与封装}

\section{类和对象}
类 (class) 是最基础也是最重要的 C\# 类型。类是一个数据结构,将状态(字段)和
操作(方法和其他函数成员)组合在一个单元中。类为动态创建类的实例 (instance) 提供定义,
实例也称为对象 (object)。类支持继承 (inheritance) 和多态性 (polymorphism),
通过继承产生派生类 (derived class), 从而扩展和专用化基类 (base class) 。

使用类声明可以创建新的类。类声明以一个声明头开始,其组成方式如下:
先指定类的特性和修饰符,后是类的名称,接着是基类(如有)以及该类实现的接口。
声明头后面跟着类体,它由一组位于一对大括号 { 和 } 之间的成员声明组成。
下面是一个名为 Point 的简单类的声明:

 \begin{lstlisting}[language=C]
public class Point
{
    public int x, y;
    public Point(int x, int y) {
        this.x = x;
        this.y = y;
    }
}
\end{lstlisting}

类的实例使用 new 运算符创建,该运算符为新的实例分配内存、
调用构造函数初始化该实例,并返回对该实例的引用。
下面的语句创建两个 Point 对象,并将对这两个对象的引用存储在两个变量中:

 \begin{lstlisting}[language=C]
Point p1 = new Point(0, 0);
Point p2 = new Point(10, 20);
\end{lstlisting}

当不再使用对象时,该对象占用的内存将自动收回。在 C\# 中,
没有必要也不可能显式释放分配给对象的内存。

\subsection{成员}
类的成员或者是静态成员 (static member),或者是实例成员 (instance member)。
静态成员属于类,实例成员属于对象(类的实例)。

\begin{tabular}{|l|l|}
\hline
成员 &  说明   \\
\hline
常量   &         与类关联的常量值 \\
字段   &         类的变量 \\
方法   &         类可执行的计算和操作 \\
属性   &         与读写类的命名属性相关联的操作 \\
索引器  &      与以数组方式索引类的实例相关联的操作 \\
事件   &         可由类生成的通知 \\
运算符  &      类所支持的转换和表达式运算符 \\
构造函数 &   初始化类的实例或类本身所需的操作 \\
析构函数 &   在永久丢弃类的实例之前执行的操作 \\
类型   &         类所声明的嵌套类型 \\
\hline
\end{tabular}

\subsection{可访问性}
类的每个成员都有关联的可访问性,它控制能够访问该成员的程序文本区域。有五种可能的可访问性形式。
下表概述了这些可访问性。

\begin{tabular}{|l|l|}
\hline
可访问性  &   含义  \\
\hline
public        &  访问不受限制   \\
protected  &  访问仅限于此类或从此类派生的类  \\
internal    & 访问仅限于此程序  \\
protected internal  & 访问仅限于此程序或从此类派生的类  \\
private & 访问仅限于此类  \\
\hline
\end{tabular}

\subsection{ 类型形参}

类定义可以通过在类名后添加用尖括号括起来的类型参数名称列表来指定一组类型参数。
类型参数可用于在类声明体中定义类的成员。在下面的示例中,Pair 的类型参数为 TFirst 和 TSecond:

 \begin{lstlisting}[language=C]
public class Pair<TFirst,TSecond>
{
    public TFirst First;
    public TSecond Second;
}
 \end{lstlisting}

要声明为采用类型参数的类类型称为泛型类类型。结构类型、接口类型和委托类型也可以是泛型。
当使用泛型类时,必须为每个类型参数提供类型实参:

 \begin{lstlisting}[language=C]
Pair<int,string> pair = new Pair<int,string> { First = 1, Second = “two” };
int i = pair.First;     // TFirst is int
string s = pair.Second; // TSecond is string
 \end{lstlisting}

提供了类型实参的泛型类型(例如上面的 Pair<int,string>)称为构造的类型。

\subsection{字段}
字段是与类或类的实例关联的变量。
使用 static 修饰符声明的字段定义了一个静态字段 (static field)。
一个静态字段只标识一个存储位置。无论对一个类创建多少个实例,
它的静态字段永远都只有一个副本。
不使用 static 修饰符声明的字段定义了一个实例字段 (instance field)。
类的每个实例都为该类的所有实例字段包含一个单独副本。
在下面的示例中,Color 类的每个实例都有实例字段 r、g 和 b 的单独副本,
但是 Black、White、Red、Green 和 Blue 静态字段只存在一个副本:

 \begin{lstlisting}[language=C]
public class Color
{
    public static readonly Color Black = new Color(0, 0, 0);
    public static readonly Color White = new Color(255, 255, 255);
    public static readonly Color Red = new Color(255, 0, 0);
    public static readonly Color Green = new Color(0, 255, 0);
    public static readonly Color Blue = new Color(0, 0, 255);

    private byte r, g, b;

    public Color(byte r, byte g, byte b) {
        this.r = r;
        this.g = g;
        this.b = b;
    }
}
 \end{lstlisting}

如上示例所示,可以使用 readonly 修饰符声明只读字段 (read-only field)。给 readonly 字段的赋值只能作为字段声明的组成部分出现,或在同一个类中的构造函数中出现。

\subsection{方法}

方法 (method) 是一种成员,用于实现可由对象或类执行的计算或操作。
静态方法 (static method) 通过类来访问。实例方法 (instance method) 通过类的实例来访问。
方法具有一个参数 (parameter) 列表(可以为空),表示传递给该方法的值或变量引用;
方法还具有一个返回类型 (return type),指定该方法计算和返回的值的类型。
如果方法不返回值,则其返回类型为 void。
与类型一样,方法也可以有一组类型参数,当调用方法时必须为类型参数指定类型实参。
与类型不同的是,类型实参经常可以从方法调用的实参推断出,而无需显式指定。
方法的签名 (signature) 在声明该方法的类中必须唯一。方法的签名由方法的名称、
类型参数的数目以及该方法的参数的数目、修饰符和类型组成。方法的签名不包含返回类型。

\subsubsection{参数}
参数用于向方法传递值或变量引用。方法的参数从调用该方法时指定的实参 (argument) 获取它们的实际值。
有四类参数:值参数、引用参数、输出参数和参数数组。
值参数 (value parameter) 用于传递输入参数。一个值参数相当于一个局部变量,
只是它的初始值来自为该形参传递的实参。对值参数的修改不影响为该形参传递的实参。
值参数可以是可选的,通过指定默认值可以省略对应的实参。
引用参数 (reference parameter) 用于传递输入和输出参数。
为引用参数传递的实参必须是变量,并且在方法执行期间,
引用参数与实参变量表示同一存储位置。引用参数使用 ref 修饰符声明。
下面的示例演示 ref 参数的用法。

 \begin{lstlisting}[language=C]
using System;
class Test
{
    static void Swap(ref int x, ref int y) {
        int temp = x;
        x = y;
        y = temp;
    }
    static void Main() {
        int i = 1, j = 2;
        Swap(ref i, ref j);
        Console.WriteLine("{0} {1}", i, j);         // Outputs "2 1"
    }
}
 \end{lstlisting}

输出参数 (output parameter) 用于传递输出参数。对于输出参数来说,
调用方提供的实参的初始值并不重要。除此之外,输出参数与引用参数类似。
输出参数是用 out 修饰符声明的。下面的示例演示 out 参数的用法。

 \begin{lstlisting}[language=C]
using System;
class Test
{
    static void Divide(int x, int y, out int result, out int remainder) {
        result = x / y;
        remainder = x % y;
    }
    static void Main() {
        int res, rem;
        Divide(10, 3, out res, out rem);
        Console.WriteLine("{0} {1}", res, rem); // Outputs "3 1"
    }
}
 \end{lstlisting}

参数数组 (parameter array) 允许向方法传递可变数量的实参。
参数数组使用 params 修饰符声明。只有方法的最后一个参数才可以是参数数组,
并且参数数组的类型必须是一维数组类型。
System.Console 类的 Write 和 WriteLine 方法就是参数数组用法的很好示例。
它们的声明如下。

 \begin{lstlisting}[language=C]
public class Console
{
    public static void Write(string fmt, params object[] args) {...}
    public static void WriteLine(string fmt, params object[] args) {...}
    ...
}
 \end{lstlisting}

在使用参数数组的方法中,参数数组的行为完全就像常规的数组类型参数。
但是,在具有参数数组的方法的调用中,既可以传递参数数组类型的单个实参,
也可以传递参数数组的元素类型的任意数目的实参。在后一种情况下,
将自动创建一个数组实例,并使用给定的实参对它进行初始化。
示例:

 \begin{lstlisting}[language=C]
Console.WriteLine("x={0} y={1} z={2}", x, y, z);
 \end{lstlisting}

等价于以下语句:

 \begin{lstlisting}[language=C]
string s = "x={0} y={1} z={2}";
object[] args = new object[3];
args[0] = x;
args[1] = y;
args[2] = z;
Console.WriteLine(s, args);
 \end{lstlisting}

\subsection{静态方法和实例方法}
使用 static 修饰符声明的方法为静态方法 (static method)。

静态方法不对特定实例进行操作,并且只能直接访问静态成员。

不使用 static 修饰符声明的方法为实例方法 (instance method)。

实例方法对特定实例进行操作,并且能够访问静态成员和实例成员。

在调用实例方法的实例上,可以通过 this 显式地访问该实例。

在静态方法中引用 this 是错误的,只能通过类名进行引用。

下面的 Entity 类具有静态成员和实例成员。

 \begin{lstlisting}[language=C]
class Entity
{
    static int nextSerialNo;

    int serialNo;

    public Entity() {
        serialNo = nextSerialNo++;
    }

    public int GetSerialNo() {
        return serialNo;
    }

    public static int GetNextSerialNo() {
        return nextSerialNo;
    }

    public static void SetNextSerialNo(int value) {
        nextSerialNo = value;
    }
}
 \end{lstlisting}

每个 Entity 实例都包含一个序号(我们假定这里省略了一些其他信息)。
Entity 构造函数(类似于实例方法)使用下一个可用的序号来初始化新的实例。
由于该构造函数是一个实例成员,它既可以访问 serialNo 实例字段,
也可以访问 nextSerialNo 静态字段。

GetNextSerialNo 和 SetNextSerialNo 静态方法可以访问 nextSerialNo 静态字段
,但是如果直接访问 serialNo 实例字段就会产生错误。

下面的示例演示 Entity 类的使用。

 \begin{lstlisting}[language=C]
using System;
class Test
{
    static void Main() {
        Entity.SetNextSerialNo(1000);
        Entity e1 = new Entity();
        Entity e2 = new Entity();
        Console.WriteLine(e1.GetSerialNo());                // Outputs "1000"
        Console.WriteLine(e2.GetSerialNo());                // Outputs "1001"
        Console.WriteLine(Entity.GetNextSerialNo());        // Outputs "1002"
    }
}
 \end{lstlisting}

注意:SetNextSerialNo 和 GetNextSerialNo 静态方法是在类上调用的,而 GetSerialNo 实例方法是在该类的实例上调用的。


\subsection{构造函数}
C\# 支持两种构造函数:实例构造函数和静态构造函数。
实例构造函数 (instance constructor) 是实现初始化类实例所需操作的成员。
静态构造函数 (static constructor) 是一种用于在第一次加载类本身时实现其初始化所需操作的成员。

构造函数的声明如同方法一样,不过它没有返回类型,并且它的名称与其所属的类的名称相同。
如果构造函数声明包含 static 修饰符,则它声明了一个静态构造函数。
否则,它声明的是一个实例构造函数。
实例构造函数可以被重载。例如,List<T> 类声明了两个实例构造函数,
一个无参数,另一个接受一个 int 参数。实例构造函数使用 new 运算符进行调用。
下面的语句分别使用 List<string> 类的每个构造函数分配两个 List 实例。

 \begin{lstlisting}[language=C]
List<string> list1 = new List<string>();
List<string> list2 = new List<string>(10);
 \end{lstlisting}

实例构造函数不同于其他成员,它是不能被继承的。一个类除了其中实际声明的实例构造函数外,
没有其他的实例构造函数。如果没有为某个类提供任何实例构造函数,
则将自动提供一个不带参数的空的实例构造函数。

\subsection{属性}

属性 (property) 是字段的自然扩展。

属性和字段都是命名的成员,
都具有相关的类型,且用于访问字段和属性的语法也相同。

然而,与字段不同,属性不表示存储位置。相反,属性有访问器 (accessor),
这些访问器指定在它们的值被读取或写入时需执行的语句。

属性的声明与字段类似,不同的是属性声明以位于定界符 { 和 } 之间的一个 get 访问器和/或一个 set 访问器结束,而不是以分号结束。

同时具有 get 访问器和 set 访问器的属性是读写属性 (read-write property),只有 get 访问器的属性是只读属性 (read-only property),只有 set 访问器的属性是只写属性 (write-only property)。
get 访问器相当于一个具有属性类型返回值的无形参方法。除了作为赋值的目标,当在表达式中引用属性时,将调用该属性的 get 访问器以计算该属性的值。
set 访问器相当于具有一个名为 value 的参数并且没有返回类型的方法。当某个属性作为赋值的目标被引用,或者作为 ++ 或 -- 的操作数被引用时,将调用 set 访问器,并传入提供新值的实参。

List<T> 类声明了两个属性 Count和 Capacity,它们分别是只读属性和读写属性。
下面是这些属性的使用示例。

 \begin{lstlisting}[language=C]
List<string> names = new List<string>();
names.Capacity = 100;           // Invokes set accessor
int i = names.Count;            // Invokes get accessor
int j = names.Capacity;         // Invokes get accessor
 \end{lstlisting}

与字段和方法相似,C\# 同时支持实例属性和静态属性。静态属性使用 static 修饰符声明,而实例属性的声明不带该修饰符。
属性的访问器可以是虚的。当属性声明包括 virtual、abstract 或 override 修饰符时,
修饰符应用于该属性的访问器。


 %%第3章 C\#面向对象特性-封装
%!TEX root = ../clcxsj.tex

\chapter{C\#面向对象特性-继承}
\section{111111}
本章主要讲述C语言与C++的不同之处,学习完本章后,应该能够在 Visual C++
编译器下基本能够正确的编译C语言程序和简单的具有C++特征的程序。

 
 %%第4章 C\#面向对象特性-继承
\include{chapter/chapter05} %%第5章 C\#面向对象特性-多态
\include{chapter/chapter06} %%第6章 C\#图形界面-WPF
%!TEX root = ../clcxsj.tex

\chapter{常用测量函数设计}

C\# 是纯面向对象语言,也就是说所有的常量与方法都需要以类$class$为载体,如同其他的数学软件包一样,
我将其命名为$SMath(SurveyMath)$,保存在$SMath.cs$文件中,为了引用方便,我们需将常用的一些常量及方法定义为静态成员。
示例代码如下所示:
\begin{verbatim}
namespace ZXY
{
    public static class SMath
    {
        public const double PI=3.1415926535897932384626433832795;
        public const double PI2=6.283185307179586476925286766559;
        public const double TODEG=57.295779513082320876798154814105;
        public const double TORAD=0.01745329251994329576923690768489;
        public const double TOSECOND=206264.80624709635515647335733078;
    
        public static double DMS2RAD(double dmsAngle)
        {
            ......
        }
    
        public static double RAD2DMS(double radAngle)
        {
            ......
        }
    }
}
\end{verbatim}

以上示例代码为了与其他的函数或符号相区别,也为了与其他的代码一起合作使用,
在此加入了自己的命名空间 ZXY (这是我用的名称空间,
你当然也可以根据自己的习惯或爱好命名适当的名称空间)。

\section{常用测量计算公式}
角度、距离与高差是测量工程师工作的基本对象,度分秒形式的角度与弧度之间的转换是我们进行测量数据处理的基础。
为了方便的进行测量数据处理,我们需要定义一些常量,如前述示例代码所示。

$PI$表示$\pi$, $PI2$表示$2\pi$;
$TODEG$表示$180/\pi$,用于将弧度化为度;
$TORAD$表示$\pi/180$,用于将度化为弧度;
$TOSECOND$表示$180*3600/\pi$,用于将弧度转换为秒。

\subsection{角度弧度互换函数:}

 在测量工程中角度的常用习惯表示法是度分秒的形式,在计算程序中测量工程人员也常将度分秒形式的角度用格式为xxx.xxxxx的形式表示,
 即以小数点前的整数部分表示度,小数点后两位数表示分,从小
 数点后第三位起表示秒。在计算机编程时所用的角度要以弧度表示的,因此需要
 设计函数相互转换。

1.角度化弧度函数:

角度化弧度函数的逻辑非常简单,许多测量编程人员将其写成如下的形式:
\begin{verbatim}
public static double DMS2RAD(double dmsAngle)
{
   int d = (int)dmsAngle;
   dmsAngle = (dmsAngle - d) * 100.0;
   int m = (int)dmsAngle;
   double s = (dmsAngle - m) * 100.0;
   return (d + m / 60.0 + s / 3600.0) * TORAD;
}
\end{verbatim}
但由于计算机中浮点数的表示方法的原因,以上函数并不能精确的将度分秒的角度转换为弧度。

 如某一角度为$1°40'00''$,我们以1.4000形式的浮点数输入,计算机将表示为1.3999999999999999的形式。
 这在计算机中并没有什么错误,但以上函数在提取角度的分秒时,提取到的m值为39,提取到的s值为
 99.999999999999,即我们提取到的角度为$1°39'100''$,有40"的角度误差,这是我们测绘工程人员不能接受的。
 
 有的软件设计人员在软件中发现这个问题后的处理的办法是让用户在角度后加减一秒,进行规避这种误差,显然,
 将软件人员的责任推给用户是极其不合适和不负责任的。还有许多的书籍中介绍了许多五花八门的处理方法,但奏效的小,
 不奏效的却很多,甚至有的将这么简单的算法逻辑变得逻辑十分复杂。
 
 虽然浮点数的表达不够精确,但我们知道在计算机中,整数的表达与计算却是精确的,因此在角度的度分秒值提取中,
 我们采用整数的运算方式进行计算,相应代码如下。
 
\begin{verbatim}
public static double DMS2RAD(double dmsAngle)
{
    dmsAngle *= 10000; 
    int angle = (int)Math.Round(dmsAngle);
    int d = angle / 10000;
    angle -= d * 10000;
    int m = angle / 100;
    double s = dmsAngle - d * 10000 - m * 100;
    return (d + m / 60.0 + s / 3600.0) * TORAD;
}
\end{verbatim}

 以上算法对于负的角度值的提取同样有效。


2.弧度化角度函数

同样的道理,以下函数将不能正确的将一些弧度值转换为度分秒形式的角度值,在某些情况下转换出的角度将出现$59'60''$或$59'59.9999''$的形式。

\begin{verbatim}
public static double RAD2DMS(double radAngle)
{
    radAngle *= TODEG;
    int d = (int)radAngle;
    radAngle = (radAngle - d) * 60;
    int m = (int)radAngle;
    double s = (radAngle - m) * 60;
    return (d + m / 100.0 + s / 10000.0);
}
\end{verbatim}

同样,我们需要利用整数的精确表达能力与计算能力来进行转换,正确的代码如下:

\begin{verbatim}
public static double RAD2DMS(double radAngle)
{
    radAngle *= TOSECOND;
    int angle = (int)Math.Round(radAngle); 
    int d = angle / 3600;
    angle -= d * 3600;
    int m = angle / 60;
    double s = radAngle - d * 3600 - m * 60;
    return (d + m / 100.0 + s / 10000.0);
}
\end{verbatim}

\subsection{坐标方位角推算}

 1.已知01边的坐标方位角$\alpha_0$和01边和12边间的水平角$\beta$,计算12边的坐标方位角。
 \begin{verbatim}
 double Azimuth(double azimuth0, double angle)
 {
    return To0_2PI(azimuth0 + angle + _PI);
 }
\end{verbatim}

 2.将角度规划到0~2π,单位为弧度
 \begin{verbatim}
 double To0_2PI(double rad)
 {
   int f = rad >= 0 ? 0 : 1;
   int n = (int)(rad / TWO_PI);

   return rad - n * TWO_PI + f * TWO_PI;
 }
 \end{verbatim}

\subsection{平面坐标正反算}
1.坐标正算:

根据0->1点的坐标方位角和水平边长,计算0->1点的坐标增量。
\begin{verbatim}
void double dxdy(double azimuth, double distance,
                 double& dx, double& dy)
{
    dx = cos(azimuth) * distance;
    dy = sin(azimuth) * distance;
}
\end{verbatim}

根据0点的坐标和0->i点的坐标方位角和水平边长,计算i点的坐标。
\begin{verbatim}
void Coordinate(double x0, double y0,
        double azimuth, double distance,
        double& xi, double& yi)
{
    xi = x0 + cos(azimuth) * distance;
    yi = y0 + sin(azimuth) * distance;
}
\end{verbatim}

根据1点的坐标和后视边(0->1)点的坐标方位角,水平角(0-1-i),水平边长(1-i),计算i点的坐标。
\begin{verbatim}
void Coordinate(double x0, double y0, double azimuth0,
        double angle, double distance,
        double& xi, double& yi)
{
    double azimuthi = Azimuth(azimuth0, angle);
    xi = x0 + cos(azimuthi) * distance;
    yi = y0 + sin(azimuthi) * distance;
}
\end{verbatim}

2.坐标反算:

计算0点至1点的坐标方位角,返回值单位为弧度。
\begin{verbatim}
double Azimuth(double x0, double y0, double x1, double y1)
{
   double dx = x1 - x0;
   double dy = y1 - y0;
   return atan2(dy, dx) + (dy < 0 ? 1 : 0) * _2PI;
}
\end{verbatim}

计算两点间(0->1)的平距
\begin{verbatim}
double Distance(double x0, double y0, double x1, double y1)
{
    double dx = x1 - x0;
    double dy = y1 - y0;
    return sqrt(dx * dx + dy * dy);
}
\end{verbatim}
 %%第7章 常用测量函数设计
%!TEX root = ../clcxsj.tex

\chapter{高斯投影正反算与换带}
\section{数据模型}
\subsection{投影换带的目的}
高斯投影是为解决球面与平面之间坐标映射问题的,即大地坐标(B,
L)与高斯平面直角坐标$(x,y)$之间的换算,以及不同带之间的高斯坐标
的换算问题。

本章将运用$C\#$编程语言编写一通用的高斯投影程序,用于1954北京坐标系、1980西安坐标系、
WGS84坐标系以及CGCS2000大地坐标系。

\subsection{高斯投影的主要内容}
\begin{enumerate}
    \item 坐标正算:将点的大地坐标转换成高斯投影平面直角坐标。
    \item 坐标反算:将点的高斯投影平面直角坐标转换成大地坐标。
    \item 换带计算:将某带的点的高斯投影平面直角坐标转换成邻带或某中央
      子午线经度的高斯投影平面直角坐标。
    \item 其他计算:计算子午线收敛角、长度比等。 
\end{enumerate}

\subsection{高斯投影的数学模型}

本章所引用的公式来自:
孔祥元,郭际明,刘宗泉.大地测量学基础-2版.武汉:武汉大学出版社,2010.5,
以下简称为大地测量学基础。

高斯投影是在椭球参数(长半轴a、短半轴b、扁率$\alpha$)一定的条件下,根据
给定的数学模型来进行计算的,我们首先分析这些计算公式。

\begin{enumerate}
 \item 基本公式

扁率:
$$\alpha=\frac{a-b}{a}$$

第一偏心率:
$$e=\sqrt{\frac{a^2-b^2}{a^2}}$$

第二偏心率:
$$e'=\sqrt{\frac{a^{2}-b^{2}}{b^{2}}}$$

子午圈曲率半径:
$$M=a(1-e^2)(1-e^2\sin ^2 B)^{-\frac{3}{2}}$$

卯酉圈曲率半径:
$$N=a(1-e^2\sin ^2 B)^{-\frac{1}{2}}$$
辅助符号:
$$t=\tan B\qquad\eta=e'\cos B$$

\item 子午线弧长
$$X=a_0 B - \frac{a_2}{2}\sin 2B + \frac{a_4}{4}\sin 4B 
- \frac{a_6}{6} \sin 6B  + \frac{a_8}{8}\sin 8B$$

式中:
\[
\left \{ \begin{aligned}
a_0 &= m_0 + \frac{m_2}{2} + \frac{3}{8}m_4 + \frac{5}{16}m_6 + \frac{35}{128}m_8  \\
a_2 &= \frac{m_2}{2} + \frac{m_4}{2} + \frac{15}{32}m_6 + \frac{7}{16}m_8  \\
a_4 &= \frac{m_4}{8} + \frac{3}{16}m_6 + \frac{7}{32}m_8  \\
a_6 &= \frac{m_6}{32} + \frac{m_8}{16}  \\
a_8 &= \frac{m_8}{128}
\end{aligned} \right.
\]

式中$m_0, m_2, m_4, m_6, m_8$的值为:
\[
\left \{ \begin{aligned}
m_0 &= a(1-e^2) \\
m_2 &= \frac{3}{2}e^2 m_0  \\
m_4 &= \frac{5}{4}e^2 m_2   \\
m_6 &= \frac{7}{6}e^2 m_4   \\
m_8 &= \frac{9}{8}e^2 m_6 
\end{aligned} \right.
\]

公式引用自《大地测量学基础》第115页(4-101)、(4-100)与(4-72).


\item 坐标正算
\[
\left \{ \begin{aligned}
x&=X+\frac{N}{2}sinBcosBl^2 +\frac{N}{24}sinBcos^3B(5-t^2 +9\eta^2+4\eta^4)l^4 \\
  &+\frac{N}{720}sinBcos^5 B(61-58t^2 +t^4)l^6  \\
y&=NcosBl+\frac{N}{6}cos^3 B(1-t^2 +\eta^2 )l^3 \\
        &+\frac{N}{120}cos^5 B (5-18t^2+t^4 +14\eta^2 -58\eta^2t^2)l^5
\end{aligned} \right.
\]


公式引用自《大地测量学基础》第169页(4-367)。

\item 坐标反算
\[
\left \{ \begin{aligned}
B&=B_f - \frac{t_f}{2M_f N_f }y^2 +\frac{t_f}{24 M_f N_f ^3}
(5 + 3t_f ^2  + \eta_f ^2 - 9\eta_f ^2 t_f^2)y^4 \\
 &- \frac{t_f}{720 M_f N_f ^5}(61 + 90t_f ^2 + 45t_f ^4)y^6 \\
l&=\frac{1}{N_f cosB_f}y - \frac{1}{6N_f ^3 cosB_f}(1 + 2t_f ^2 + \eta_f ^2)y^3  \\
 &+ \frac{1}{120N_f ^5 cosB_f}(5 + 28t_f ^2 + 24t_f ^4 + 6\eta_f ^2 +8\eta_f ^2 t_f ^2)y^5
\end{aligned} \right.
\]

公式引用自《大地测量学基础》第171页(4-383)。


\item 平面子午线收敛角计算

利用$(B, l)$计算公式如下:
$$\gamma = \sin B \cdot l + \frac{1 + 3 \eta^2 + 2 \eta^4}{3} \sin B \cos ^2 B \cdot l^3 
+ \frac{2 - t^2}{15}\sin B \cos ^4 B \cdot l^5$$

公式引用自《大地测量学基础》第181页(4-408)。

利用$(x, y)$计算公式如下:

$$\gamma = \frac{1}{N_f}t_f y - \frac{1+t_f ^2 - \eta_f ^2}{3N_f ^3}t_f y^3 
+ \frac{2+5t_f^2+3t_f^4}{15N_f ^5}t_fy^5$$

公式引用自《大地测量学基础》第182页(4-410)。

\item 长度比计算

利用$(B, l)$计算公式如下:
$$m=1+\frac{1}{2}l^2 \cos ^2 B(1+\eta^2) + \frac{1}{24}l^4\cos ^4 B(5-4t^2)$$

公式引用自《大地测量学基础》第189页(4-447)。

利用$(x, y)$计算公式如下:

$$m=1+\frac{y^2}{2R^2} + \frac{y^4}{24R^4}$$

式中$R=\sqrt{MN}$,公式引用自《大地测量学基础》第189页(4-451)。

\end{enumerate}

\subsection{数学模型分析}
分析以上各个计算公式发现,如果长半轴与扁率确定,参考椭球的第一偏心率$e$、
第二偏心率$e'$,辅助计算参数$(m_0, m_2, m_4, m_6, m_8)$与
$(a_0, a_2, a_4, a_6, a_8)$也就确定了。也就是说这些参数对于某一种确定的
参考椭球是常数。而$(M,N,t,\eta)$则是纬度$B$的函数。

\section{高斯投影基本功能实现}

为了让我们的算法能被其他项目使用,我们在此新建一 Class Library(.NET Framework) 项目,
在项目中将以前所写的 SMath.cs 文件加入,将命名空间
改为 ZXY 。

基于以上分析,我们新建一椭球类$(Ellipsoid)$,在其中我们定义以上与椭球类型
相关的元素。代码如下所示:
\begin{verbatim}
//File Ellipsoid.cs
using System;

namespace ZXY
{
    public class Ellipsoid
    {
        private double a;
        private double b;
        private double f; //(a-b)/a = 1/f

        private double e2;
        private double eT2;

        private double a0;
        private double a2;
        private double a4;
        private double a6;
        private double a8;

        private double m0;
        private double m2;
        private double m4;
        private double m6;
        private double m8;

        .................................
    }
}
\end{verbatim}

C语言是典型的面向过程设计语言,小巧、灵活,功能强大。
下面我们将从一个小功能开始构造满足以上全部功能的程序。

在此我们先完成正算功能,我们设定正算中所用的角度均为弧度,从而避免在正算中进行
度分秒与弧度的转换问题,我们也先把读数据文件、多点等问题去掉,从只计算
一个点开始考虑问题。我们现在只考虑54北京坐标系的问题,别的暂且不管。假
如我们的主函数的调用形式如下:
\begin{verbatim}
void main()
{
    //克拉索夫斯基参考椭球
    double a = 6378245.0,  f = 298.3;
    //正算点的纬度和经差
    double B = 0.3836189311, l = 0.0423314484;
    // B = 21°58′47.0845″ l = 2°25′31.4880″
    double x, y;
    CalEcd(a, f);
    GaussZs(B, l, &x, &y);
    printf("x = %lf, y = %lf \n", x, y);
    //算出的x=2433586.692, y=250547.403
}
\end{verbatim}

以上程序的流程为:

(1)给定一个类型的参考椭球:如长半轴a, 扁率f

(2)计算相关的参数:计算偏心率,子午线弧长系数等,这些相对于一个给定
的椭球来说,它是一个定值;

(3)根据给定的B,l 计算相应的高斯平面坐标;

(4)输出计算值。

从流程上可看出,GaussZs函数的原型如下:
\begin{verbatim}
void GaussZs(double B, double l,
             double * x, double * y)
\end{verbatim}
这里B,l为给定值,x,y设定为指针类型,用于返回计算后的值。下面我们来实
现这个函数,它的计算流程如下:

(1)计算子午线弧长;

(2)计算x和y坐标;

我们设计该函数如下:
\begin{verbatim}
void GaussZs(double B, double l,
             double * x, double * y)
{
    double sinB = sin( B );
    double cosB = cos( B );
    double cosB2 = cosB * cosB;
    double cosB4 = cosB2 * cosB2;
    double l2 = l * l;
    double l4 = l2 * l2;

    double g = _eT * cosB;
    double g2 = g * g;
    double g4 = g2 * g2;
    double t = tan(B);
    double t2 = t * t;
    double t4 = t2 * t2;

    //计算子午线弧长
    double X = MeridianArcLength( B );
    double N = _c / sqrt(1.0 + g2);

    *x = X + N * sinB * cosB * l2 *
    ( 0.5
      + cosB2 * l2 * (5.0 - t2 + 9.0 * g2 + 4.0 * g4) / 24.0
      + cosB4 * l4 * (61.0 - 58.0 * t2
          + t4 + 270.0 * g2 - 330.0 * g2 * t2)  / 720.0
    );
    *y = N * cosB * l *
    ( 1.0
      + cosB2 * l2 * (1.0 - t2 + g2) / 6.0
      + cosB4 * l4 * (5.0 - 18.0 * t2 + t4
          + 14.0 * g2 - 58.0 * t2 * g2) / 120.0
    );
}
\end{verbatim}

从函数的实现看,前面的变量只是为了简化后面的计算式,整个函数中没有逻辑
判断和循环,因此较为简单。但子午线弧长的计算函数MeridianArcLength(
double )我们还没实现,根据算法,它的实现如下:
\begin{verbatim}
double MeridianArcLength(double B)
{
    double sinB = sin(B);
    double cosB = cos(B);
    double sinB3 = sinB * sinB * sinB;
    double sinB5 = sinB * sinB * sinB3;
    double sinB7 = sinB * sinB * sinB5;
    return _A0 * B - cosB * (_B0 * sinB + _C0 * sinB3
           + _D0 * sinB5 + _E0 * sinB7);
}
\end{verbatim}
程序中的\verb|_A0、_B0、_C0、_D0、_E0|为子午线弧长计算的系数,在此我把它们设
为全局变量,在CalEcd函数中进行计算。如:

\begin{verbatim}
double _a, _b, _c, _d, _e, _eT;
double _A0, _B0, _C0 , _D0, _E0;
double _e2, _e4, _e6, _e8;
\end{verbatim}

由于这些相关的系数只与椭球的参数有关,我们只在函数CalEcd计算一次。其实
现如下:
\begin{verbatim}
void CalEcd(double a, double f)
{
    _a = a; _b = _a * (1.0 - 1.0 / f);

    _e = sqrt(_a * _a - _b * _b) / _a;
    _eT = sqrt(_a * _a - _b * _b) / _b;
    _c = _a / _b * _a;
    _d = _b / _a * _b;

    _e2 = _e * _e;
    _e4 = _e2 * _e2;
    _e6 = _e2 * _e4;
    _e8 = _e2 * _e6;

    _A0 = _d * (  1    + 3.0 /4.0 * _e2
                   + 45.0 / 64.0 * _e4
               + 175.0 / 256.0 * _e6
               + 11025.0 / 16384.0 * _e8 );
    _B0 = _d * (         3.0 /4.0 * _e2
                   + 45.0 / 64.0 * _e4
               + 175.0 / 256.0 * _e6
               + 11025.0 / 16384.0 * _e8);
    _C0 = _d * (         15.0 / 32.0 * _e4
               + 175.0 / 384.0 * _e6
                       + 3675.0 /  8192.0 * _e8);
    _D0 = _d * (         35.0 /  96.0 * _e6
                   + 735.0 /  2048.0 * _e8);
    _E0 = _d * (     315.0 /  1024.0 * _e8 );
}
\end{verbatim}

至此,我们完成了高斯平面直角坐标的正算功能的程序设计。为了利用C++语言
增强了的C功能,我们把文件的扩展名由c改为cpp,以利用C++编译器进行严格检
查。

为了进行源代码一级的复用,我们在工程中新加入一个头文件(.h),将上面除
主函数main()之外的所有代码剪切到头文件中(如头文件为Gauss.h),在主函
数所在的文件中将头文件包含在内,即在文件顶端加入:\verb|#include "Gauss.h"|,
在别的地方我们就可继续使用这些函数。

为了直接利用纬度、经度和中央子午线的经度计算高斯平面坐标,我们可以利用函数重
载的功能继续实现坐标正算:

\begin{verbatim}
void GaussZs(double B, double L, double L0,
             double * x, double * y)
{
    double l = L - L0;
    GaussZs(B, l, x, y);
}
\end{verbatim}

这里的实现很简单,只是计算了经差,调用了原来的正算函数。我们就可在主函
数中作如下形式的调用了:

\begin{verbatim}
void main()
{
    //21°58′47.0845″的弧度形式
    double B = 0.3836189311;
    // 113°25′31.4880″的弧度形式
    double L = 1.9796469181;
    // 111°的弧度形式
    double L0 = 1.9373154697;
    //54//x:2433586.692, y:250547.403
    GaussZs(B, L, L0, &x, &y);
    printf("x = %lf, y = %lf \n", x, y);
}
\end{verbatim}
至此,我们已经完成了高斯坐标正算的全部计算了。以上的函数调用中的角度均
使用了弧度的形式,利用前面所讲的角度化弧度的函数,我们可以以下形式的调
用:

\begin{verbatim}
void main()
{
    //21°58′47.0845″
    double B = DMStoRAD(21.58470845);
    //113°25′31.4880″
    double L = DMStoRAD (113.25314880);
    double L0 = DMStoRAD(111.0); //111°
    //54//x:2433586.692, y:250547.403
    GaussZs(B, L, L0, &x, &y);
    printf("x = %lf, y = %lf \n", x, y);
}
\end{verbatim}

同样,坐标反算的程序设计的实现方法也基本一样。由于面向过程的程序设计不
是我们的重点,就将这部分的实现放入面向对象中。

\section{*面向对象的高斯坐标正反算程序设计}
在前面的程序设计中,我们将椭球参数用全局变量表示,为了消除这些全局变量
的影响,我们采用类的形式对其进行封装,封装的形式如下:

\begin{verbatim}
class CEarth
{
private:
    double _a, _b, _c, _d, _e, _eT;
    double _A0, _B0, _C0 , _D0, _E0;
    double _e2, _e4, _e6, _e8, _e10;
};
\end{verbatim}

将它们设为private,防止类以外的代码随意访问它们。
再将上面实现的正算等函数作为其成员函数,则形式为:

\begin{verbatim}
class CEarth
{
private:
    double _a, _b, _c, _d, _e, _eT;
    double _A0, _B0, _C0 , _D0, _E0;
    double _e2, _e4, _e6, _e8, _e10;
public:
    CEarth();//构造函数
    virtual ~CEarth();  //析构函数
    //计算相关系数
    void CalEcd(double a, double alf);
    //正算
    void GsZs(double l, double B,
              double& x, double& y);
    void GsZs(double L, double B, double L0,
              double& x, double& y);
    //反算
    void GsFs(double x, double y,
              double& l, double& B);
    void GsFs(double x, double y, double L0,
              double & L, double & B);
    //子午线收敛角
    double MeridianConvergentAngleByBl(
              double l, double B);
    double MeridianConvergentAngleByxy(
              double x, double y);
    //子午线弧长
    double MeridianArcLength(double B);
     //底点纬度
    double Bf(double x);
};
\end{verbatim}

在类的成员函数定义中,将传进的参数用引用的形式加const进行限定,同时用
引用取代指针进行函数返回值,如GsZs,GsFs函数。

在这里,我们看看反算的实现。在反算的计算流程中,经差l的计算很简单,直
接计算即可,但纬度的计算需要先计算底点纬度,由于我们的程序是针对多种椭
球的,底点纬度的数值计算公式是不能使用的。我们用子午线弧长计算公式进行
迭代计算,在开始迭代时,取弧长的初值为\verb|x / _A0|,其具体实现如下:

\begin{verbatim}
double CEarth::Bf(double x)
{
    double Bf0 = x / _A0; //子午线弧长的初值
    int i = 0;
    while( i < 10000 )//设定迭代次数
    {
        double sinBf = sin(Bf0);
        double cosBf = cos(Bf0);
        double sinBf3 = sinBf * sinBf * sinBf;
        double sinBf5 = sinBf * sinBf * sinBf3;
        double sinBf7 = sinBf * sinBf * sinBf5;

        double Bf = (  x
                 + cosBf * ( _B0 * sinBf
                                + _C0 * sinBf3
                    + _D0 * sinBf5
                    + _E0 * sinBf7)
            ) / _A0;
        if( fabs(Bf - Bf0) < 1e-10) //计算精度
            return Bf;
        else
        {
            Bf0 = Bf;
            i++;
        }
    }
    return -1e12;
}
\end{verbatim}

反算的实现如下:

\begin{verbatim}
void CEarth::GsFs(double x, double y,
                  double & l, double & B)
{
    double Bf0 = Bf( x );
    double cosBf = cos( Bf0 );

    double gf = _eT * cosBf;
    double gf2 = gf * gf;
    double gf4 = gf2 * gf2;

    double tf = tan(Bf0);
    double tf2 = tf * tf;
    double tf4 = tf2 * tf2;

    double Nf = _c / sqrt(1.0 + gf2);
    double Nf2 = Nf * Nf;
    double Nf4 = Nf2 * Nf2;

    double y2 = y * y;
    double y4 = y2 * y2;

    l = y / (Nf * cosBf) *
    ( 1.0
      - y2 / (6.0 * Nf2) * (1.0 + 2.0 * tf2 + gf2)
          + y4 / (120.0 * Nf4 ) * (5.0
      + 28.0 * tf2
          + 24.0 * tf4
      + 6.0 * gf2
      + 8.0 * gf2 * tf2)
    );
    B = Bf0 - y2 / Nf2 * tf * 0.5 *
      ( (1.0 + gf2)
             - y2 * (5.0
                     + 3.0 * tf2
                     + 6.0 * gf2
                     - 6.0 * tf2 * gf2
                     - 3.0 * gf4
                     + 9.0 * gf4 * tf4
                    ) / (12.0 * Nf2)
             + y4 * (61.0
                     + 90.0 * tf2
                     + 45.0 * tf4
                     + 107.0 * gf2
                     + 162.0 * gf2 * tf2
                     + 45.0 * gf2 * tf4
                    ) / ( 360 * Nf4)
     );
}

//L:经度(弧度), B:纬度(弧度), L0:中央子午线经度
void CEarth::GsFs(double x, double y, double L0,
                  double& L, double& B)
{
    double l;
    GsFs(x, y, l, B);
    L = L0 + l;
}
\end{verbatim}

在以上实现中,由于我们的成员变量均为private类型,而构造函数为默认的形
式,无法将椭球的参数值传递进去。由于要实现54、80和自定义椭球的坐标系,
相应的解决方法有:

1.将默认的构造函数改为CEarth (double a, double alf)形式,将不同的椭球
参数传给类。比如要建立54坐标系,只需要如下作即可:
\begin{verbatim}
CEarth earth54(6378245.0, 298.3);
CEarth * pEarth54 = new CEarth(6378245.0, 298.3);
\end{verbatim}

这种形式很直观,但缺点也很明显,对于已知的常用椭球,编程人员每次均要提
供其具体的参数值,很不方便,也容易出错。

2.利用类厂的方法,将生成已知的常用椭球定义为静态成员函数,同时将构造
函数声明为private类型,防止象方法1那样直接访问。这种方法的优点是显而易
见的,它的实现如下:

在类CEarth中将默认的构造函数的访问域改为:
\begin{verbatim}
private:
    CEarth();
    void CalEcd(double a, double alf);
\end{verbatim}

同时CalEcd函数计算椭球内部的基本系数,不需要外部程序访问,也将其定义为
private。

同时在类中声明以静态成员指针变量:
\begin{verbatim}
private:
    static CEarth * _pEarth;
\end{verbatim}
在类的实现文件前面进行初始化:
\begin{verbatim}
CEarth* CEarth::_pEarth = NULL;
\end{verbatim}

再在类中定义用于创建椭球的静态成员函数:
\begin{verbatim}
public:
    static CEarth * CreateEarth54();
    static CEarth * CreateEarth80();
    static CEarth * CreateEarthCustomize(double a, double alf);
\end{verbatim}

它们的实现为:
\begin{verbatim}
CEarth * CEarth::CreateEarth54()
{
    if(_pEarth == NULL) _pEarth = new CEarth();
    _pEarth->CalEcd(6378245.0, 298.3);
    return _pEarth;
}
CEarth * CEarth::CreateEarth80()
{
    if(_pEarth == NULL) _pEarth = new CEarth();
    _pEarth->CalEcd(6378140.0, 298.257);
    return _pEarth;
}
CEarth * CEarth::CreateEarthCustomize(double a, double alf)
{
    if(_pEarth == NULL) _pEarth = new CEarth();
    _pEarth = new CEarth();
    _pEarth->CalEcd(a, alf);
    return _pEarth;
}
\end{verbatim}

从它们的实现看,是用new的方式生成了椭球对象,为了避免内存泄漏,类的析
构函数中,负责将申请的内存进行释放,它的实现如下:
\begin{verbatim}
CEarth::~CEarth()
{
    if(_pEarth != NULL)
    {
        delete _pEarth;
        _pEarth = NULL;
    }
}
\end{verbatim}

主函数调用的形式为:
\begin{verbatim}
void main()
{
    double x, y;
CEarth * pEarth = CEarth::CreateEarth54();
    double B = zx::CSurMath::DmsToRad(21.58470845);
    double l = zx::CSurMath::DmsToRad(113.25314880)
                   - zx::CSurMath::DmsToRad(111);
    pEarth->GsZs( l, B, x, y);
printf("x = %lf, y = %lf \n", x, y);
}
\end{verbatim}
现在我们基本上用面向对象的方法完成了高斯投影正反算的问题了。

\section{实现换带计算和文件读写}
\subsection{换带计算}
换带计算的基本方法:已知某一带(中央子午线经度已知oldL0)的点坐标
(oldX, oldY),利用高斯反算计算出大地坐标(B, L),再根据新带的中央
子午线经度(newL0),高斯正算计算新带中的高斯平面坐标(newX,  newY)。
用算法描述如下:

1.$(B,L)= GsFs(oldX, oldY, oldL0)$

2.$(newX,  newY) = GsZs(B, L, newL0)$

用一函数描述它为:
\begin{verbatim}
void CEarth::GsHd(double oldX, double oldY,
                  double oldL0,  double newL0,
                  double& newX, double& newY)
{
    double B, L;
    GsFs(oldX, oldY, oldL0, L, B);
    GsZs(L, B, newL0, newX, newY);
}
\end{verbatim}

注意:高斯换带计算一般是指同一参考椭球的不同带之间的换算,不同的参考椭
球的坐标系是不能采用这种方法的。

计算示例如下:

\begin{verbatim}
void main()
{
    double oldX = 3275110.535;
    double oldY = 235437.233;
    double oldL0 = zx::CSurMath::DmsToRad(117);
    double newL0 = zx::CSurMath::DmsToRad(120);
    double newX, newY;

   CEarth * pEarth = CEarth::CreateEarth54();
   pEarth->GsHd(oldX, oldY, oldL0, newL0, newX, newY);
   printf("newX = %lf,  newY = %lf \n", newX,  newY);
   // 3272782.315, -55299.545
}
\end{verbatim}

\subsection{多点计算和文件读写}
以上我们的算法和测试验证程序均是针对一个点而言的,如果我们在一个文本形
式的数据文件里存放有多个点,怎么计算呢?

1、我们设计正算的数据文件格式为:
\begin{verbatim}
成果文件名

转换点个数 ~ 中央子午线经度

点名 ~ 大地纬度 ~ 大地经度
\end{verbatim}

示例数据文件为:

\begin{verbatim}
BLXY.txt
3  111
p1  21.58470845  113.25314880
p2  31.04416832  111.47248974
p3  30.45254425  111.17583596
\end{verbatim}

则设计函数为:
\begin{verbatim}
struct PntInfo
{
    char name[10];
    double B, L;
    double x, y;
};
void Zs()
{
    double L0;//所在带的中央子午线经度
    int n;//转换点的个数
    char cg[256];//成果文件名
    FILE *in;
    //读取文本文件的数据
    in = fopen("BLtoXY.txt", "r"); //打开已知文件
    fscanf(in,"%s",cg); //首先读入成果文件名称
    fscanf(in,"%d %lf", &n, &L0); //转换点的个数  中央子午线经度
    PntInfo * pnts = new PntInfo[n];//动态数组
    for(int i = 0; i < n; i++)//循环读入点名B、L
        fscanf(in, "%s %lf %lf", pnts[i].name, &pnts[i].B, &pnts[i].L);
    fclose(in);
    //以下进行正算计算
    CEarth * pEarth = CEarth::CreateEarth54();
    for(i = 0; i < n; i++)
        pEarth->GsZs(zx::CSurMath::DmsToRad(pnts[i].L),
         zx::CSurMath::DmsToRad(pnts[i].B),
             zx::CSurMath::DmsToRad(L0),
         pnts[i].x, pnts[i].y);
    //将计算后的数据写到成果文件中
    FILE * out;
    out = fopen(cg, "w");
    fprintf(out, "大地坐标(B,L)=====>国家坐标(X,Y)\n ");
    fprintf(out,"中央子午线经度: L0 = %lf\n", L0);
    fprintf(out,"序号 点名 B      L ===> X坐标(m)  Y坐标(m)\n");
    for(i = 0; i < n; i++)
    {
    fprintf(out," %3d %s %lf %lf %11.3f %11.3f\n",
        i+1, pnts[i].name, pnts[i].B, pnts[i].L,
            pnts[i].x, pnts[i].y );
    }
    fclose(out);
    delete[] pnts;//释放申请的内存
}
\end{verbatim}

同样也可实现多点的数据文件反算和换带计算

反算的数据文件格式设为:XYBL.txt
\begin{verbatim}
3   111
p1     2433586.692   250547.403
p2     3439978.970    75412.872
p3     3404139.839    28680.571
\end{verbatim}

换带计算的数据文件格式设为:
\begin{verbatim}
XYBL.txt
3   111 112
p1     2433586.692   250547.403
p2     3439978.970    75412.872
p3     3404139.839    28680.571
\end{verbatim}
相应的实现请参照正算实现。

\section{小结}
我们从一个逻辑结构不太明显的程序开始,将其改为一个结构较好的面向过程的
程序。并从一个算法开始逐步实现了整个程序。最后用面向对象的方法将其改写
并实现了全部功能。最后的程序从结构上看明显要比第一个程序要好,且更易维
护和扩充。当然高斯换带计算较为简单,我们的程序实现同样也较为简单。从这
个例子,我们知道了怎样设计结构良好的程序。
 %%第8章 高斯投影程序设计
\include{chapter/chapter09} %%第9章 坐标转换程序设计
%!TEX root = ../clcxsj.tex

\chapter{线路要素计算程序设计}
\section{111111}
本章主要讲述C语言与C++的不同之处,学习完本章后,应该能够在 Visual C++
编译器下基本能够正确的编译C语言程序和简单的具有C++特征的程序。

 
 %%第10章 线路要素计算程序设计
\include{chapter/chapter11} %%第11章 单导线近似平差程序设计

  % 附录\appendix
  %\include{chapter/appendix}

\end{document}
