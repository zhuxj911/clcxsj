%!TEX root = ../clcxsj.tex

\chapter{高斯投影正反算与换带}

高斯投影是为解决球面与平面之间的坐标映射问题,即大地坐标(B,
L)与高斯平面直角坐标$(x,y)$之间的相互换算,以及不同带之间的高斯坐标
的换算问题。

本章将运用$C\#$编程语言编写一个通用的高斯投影程序,用于1954北京坐标系、1980西安坐标系、
WGS84坐标系以及CGCS2000大地坐标系或自定义参考椭球的高斯投影正反算与换带计算。


\section{高斯投影的数学模型}

为了编制正确而且高效的高斯投影与换带程序,我们首先需要分析高斯投影的数学模型,也就是我们
常说的算法。

本章所引用的公式来自:
孔祥元,郭际明,刘宗泉.大地测量学基础-2版.武汉:武汉大学出版社,2010.5,
以下将这本书简称为大地测量学基础或大地测量学。

高斯投影是在椭球的几何参数(长半轴a、短半轴b、扁率$\alpha$)确定的条件下,根据
给定的数学模型来进行计算的,我们首先分析这些计算公式与数学模型。

\begin{enumerate}

 \item 基本公式

\begin{enumerate}
\item 扁率:
$$\alpha=\frac{a-b}{a}$$

\item 第一偏心率:
$$e=\sqrt{\frac{a^2-b^2}{a^2}}$$

\item 第二偏心率:
$$e'=\sqrt{\frac{a^{2}-b^{2}}{b^{2}}}$$

\item 子午圈曲率半径:
$$M=\frac{a(1-e^2)}{(1-e^2\sin ^2 B)^{\frac{3}{2}}}$$

\item 卯酉圈曲率半径:
$$N=\frac{a}{\sqrt{1-e^2\sin^2 B}}$$

\item 辅助符号:
$$t=\tan B\qquad\eta=e'\cos B$$
\end{enumerate}

\item 椭球面梯形图幅面积计算

$$P = \frac{b^2}{2}(L_2 - L_1) \left | \frac{\sin B}{1-e^2\sin^2 B} + \frac{1}{2e}\ln \frac{1+e\sin B}{1-e\sin B}\right |_{B1} ^{B2} $$
公式引用自《大地测量学基础》第121页(4-140)。

$$P=b^2(L_2 - L_1) \left | sinB + \frac{2}{3}e^2 \sin ^3 B + \frac{3}{5}e^4 \sin^5 B + \frac{4}{7}e^6 \sin ^7 B \right | _{B_1} ^{B_2}$$

公式引用自《大地测量学基础》第121页(4-142),该公式为(4-140)的展开式。

\item 子午线弧长
$$X=a_0 B - \frac{a_2}{2}\sin 2B + \frac{a_4}{4}\sin 4B
- \frac{a_6}{6} \sin 6B  + \frac{a_8}{8}\sin 8B$$

式中:
\[
\left \{ \begin{aligned}
a_0 &= m_0 + \frac{m_2}{2} + \frac{3}{8}m_4 + \frac{5}{16}m_6 + \frac{35}{128}m_8  \\
a_2 &= \frac{m_2}{2} + \frac{m_4}{2} + \frac{15}{32}m_6 + \frac{7}{16}m_8  \\
a_4 &= \frac{m_4}{8} + \frac{3}{16}m_6 + \frac{7}{32}m_8  \\
a_6 &= \frac{m_6}{32} + \frac{m_8}{16}  \\
a_8 &= \frac{m_8}{128}
\end{aligned} \right.
\]

式中$m_0, m_2, m_4, m_6, m_8$的值为:
\[
\left \{ \begin{aligned}
m_0 &= a(1-e^2) \\
m_2 &= \frac{3}{2}e^2 m_0  \\
m_4 &= \frac{5}{4}e^2 m_2   \\
m_6 &= \frac{7}{6}e^2 m_4   \\
m_8 &= \frac{9}{8}e^2 m_6
\end{aligned} \right.
\]

公式引用自《大地测量学基础》第115页(4-101)、(4-100)与(4-72).


\item 坐标正算
\[
\left \{ \begin{aligned}
x&=X+\frac{N}{2}sinBcosBl^2 +\frac{N}{24}sinBcos^3B(5-t^2 +9\eta^2+4\eta^4)l^4 \\
  &+\frac{N}{720}sinBcos^5 B(61-58t^2 +t^4)l^6  \\
y&=NcosBl+\frac{N}{6}cos^3 B(1-t^2 +\eta^2 )l^3 \\
        &+\frac{N}{120}cos^5 B (5-18t^2+t^4 +14\eta^2 -58\eta^2t^2)l^5
\end{aligned} \right.
\]


公式引用自《大地测量学基础》第169页(4-367)。

\item 坐标反算
\[
\left \{ \begin{aligned}
B&=B_f - \frac{t_f}{2M_f N_f }y^2 +\frac{t_f}{24 M_f N_f ^3}
(5 + 3t_f ^2  + \eta_f ^2 - 9\eta_f ^2 t_f^2)y^4 \\
 &- \frac{t_f}{720 M_f N_f ^5}(61 + 90t_f ^2 + 45t_f ^4)y^6 \\
l&=\frac{1}{N_f cosB_f}y - \frac{1}{6N_f ^3 cosB_f}(1 + 2t_f ^2 + \eta_f ^2)y^3  \\
 &+ \frac{1}{120N_f ^5 cosB_f}(5 + 28t_f ^2 + 24t_f ^4 + 6\eta_f ^2 +8\eta_f ^2 t_f ^2)y^5
\end{aligned} \right.
\]

公式引用自《大地测量学基础》第171页(4-383)。


\item 平面子午线收敛角计算

利用$(B, l)$计算公式如下:
$$\gamma = \sin B \cdot l + \frac{1 + 3 \eta^2 + 2 \eta^4}{3} \sin B \cos ^2 B \cdot l^3
+ \frac{2 - t^2}{15}\sin B \cos ^4 B \cdot l^5$$

公式引用自《大地测量学基础》第181页(4-408)。

利用$(x, y)$计算公式如下:

$$\gamma = \frac{1}{N_f}t_f y - \frac{1+t_f ^2 - \eta_f ^2}{3N_f ^3}t_f y^3
+ \frac{2+5t_f^2+3t_f^4}{15N_f ^5}t_fy^5$$

公式引用自《大地测量学基础》第182页(4-410)。

\item 长度比计算

利用$(B, l)$计算公式如下:
$$m=1+\frac{1}{2}l^2 \cos ^2 B(1+\eta^2) + \frac{1}{24}l^4\cos ^4 B(5-4t^2)$$

公式引用自《大地测量学基础》第189页(4-447)。

利用$(x, y)$计算公式如下:

$$m=1+\frac{y^2}{2R^2} + \frac{y^4}{24R^4}$$

式中$R=\sqrt{MN}$,公式引用自《大地测量学基础》第189页(4-451)。

\end{enumerate}

\section{程序功能分析与设计}

\subsection{高斯投影的主要内容}
\begin{enumerate}
    \item 坐标正算:将点的大地坐标转换成高斯投影平面直角坐标。
    \item 坐标反算:将点的高斯投影平面直角坐标转换成大地坐标。
    \item 换带计算:将某带的点的高斯投影平面直角坐标转换成邻带或某中央
    子午线经度的高斯投影平面直角坐标。
    \item 其他计算:计算子午线收敛角、长度比等。
\end{enumerate}

\subsection{参考椭球类的设计}

分析以上各个计算公式发现,如果椭球长半轴与扁率确定,参考椭球的第一偏心率$e$、
第二偏心率$e'$,辅助计算参数$(m_0, m_2, m_4, m_6, m_8)$与
$(a_0, a_2, a_4, a_6, a_8)$也就确定了,其中的$(m_0, m_2, m_4, m_6, m_8)$
作为计算$(a_0, a_2, a_4, a_6, a_8)$的值使用过一次。也就是说这些参数对于某一种确定的
参考椭球是常数。而$(M,N,t,\eta)$则是纬度$B$的函数。

我们新建一 WPF 项目,命名为GaussProj,程序中我们将应用WPF技术编写图形界面。

在高斯投影中由于要处理角度与点等数据,我们可以将前面的角度处理函数DMS2RAD与RAD2DMS与SPoint点类
拷贝到当前项目中来,也可以将其打包成一 Class Library(.NET Framework)即类库文件引用到当前项目中,
在当前项目中尽量不修改前面的各个功能代码。

基于上一小节的分析,我们新建一椭球类$(Spheroid)$,在其中我们定义以上与椭球类型
相关的元素。代码如下所示:
 \begin{lstlisting}[language=C]
namespace GaussProj
{
    /// <summary>
    /// 参考椭球
    /// </summary>
    public class Spheroid
    {
        /// <summary>
        /// 长半轴
        /// </summary>
        public double a { get; set; }

        /// <summary>
        /// 短半轴
        /// </summary>
        public double b { get; set; }

        /// <summary>
        /// 扁率分母
        /// </summary>
        public double f { get; set; }  //(a-b)/a = 1/f

        /// <summary>
        /// 第一偏心率的平方
        /// </summary>
        public double e2 { get; set; }

         /// <summary>
        /// 第二偏心率的平方
        /// </summary>
        public double eT2 { get; set; }

        //计算子午线弧长时的各个系数项
        private double a0;
        private double a2;
        private double a4;
        private double a6;
        private double a8;
    }
}
\end{lstlisting}

在各项计算中,第一偏心率与第二偏心率的直接使用较少,其平方值用的较多,因此在
Spheroid类我们直接用其平方值。

 在Spheroid类中我们可以如下直接定义构造函数用于初始化其各个字段(Field):
  \begin{lstlisting}[language=C]
        /// <summary>
        /// 构造函数
        /// </summary>
        private Spheroid(double semimajor_axis, double inverse_flattening)
        {
            this.a = semimajor;
            this.f =  inverse_flattening;
            ...................................................
        }
\end{lstlisting}

但这样我们可能无法为已知的一些参考椭球直接提供参数,所以我们将构造函数定义
为 private,让用户无法通过new直接创建实例化对象。我们设计类的静态函数
Create**** 来创建类Spheroid的实例化对象,其代码如下:

 \begin{lstlisting}[language=C]
        /// <summary>
        /// 构造函数
        /// </summary>
        private Spheroid(){}

        /// <summary>
        /// 初始化椭球参数的内部函数
        /// </summary>
        private void Init(double semimajor_axis, double inverse_flattening)
        {
            this.a = semimajor_axis; this.f = inverse_flattening;
            b = a * (1 - 1 / f);

            e2 = 1 - b / a * b / a;
            eT2 = a / b * a / b - 1;

            double m0 = a * (1 - e2);
            double m2 = 3.0 / 2.0 * e2 * m0;
            double m4 = 5.0 / 4.0 * e2 * m2;
            double m6 = 7.0 / 6.0 * e2 * m4;
            double m8 = 9.0 / 8.0 * e2 * m6;

            a0 = m0 + m2 / 2.0 + 3.0 / 8.0 * m4 + 5.0 / 16.0 * m6
                          + 35.0 / 128.0 * m8;
            a2 = m2 / 2.0 + m4 / 2.0 + 15.0 / 32.0 * m6 + 7.0 / 16.0 * m8;
            a4 = m4 / 8.0 + 3.0 / 16.0 * m6 + 7.0 / 32.0 * m8;
            a6 = m6 / 32.0 + m8 / 16.0;
            a8 = m8 / 128.0;
        }

        public static Spheroid CreateBeiJing1954()
        {
            Spheroid spheroid = new Spheroid();
            spheroid.Init(6378245, 298.3);
            return spheroid;
        }

        public static Spheroid CreateXian1980()
        {
            Spheroid spheroid = new Spheroid();
            spheroid.Init(6378140, 298.257);
            return spheroid;
        }

        public static Spheroid CreateWGS1984()
        {
            Spheroid spheroid = new Spheroid();
            spheroid.Init(6378137, 298.257223563);
            return spheroid;
        }

        public static Spheroid CreateCGCS2000()
        {
            Spheroid spheroid = new Spheroid();
            spheroid.Init(6378137, 298.257222101);
            return spheroid;
        }

        public static Spheroid CreateCoordinateSystem(double semimajor_axis,
              double inverse_flattening)
        {
            Spheroid spheroid = new Spheroid();
            spheroid.Init(semimajor_axis, inverse_flattening);
            return spheroid;
        }
\end{lstlisting}

这种将构造函数设为private,通过静态函数来创建实例化对象的技术在软件设计中
会经常用到。同时在设计类的方法时,应注意访问权限的设置,如上面代码中Init函数,
在类中用于初始化椭球的各项几何参数,并不需要在类外来调用它,所以我们将其设置为
private权限。

\subsection{高斯投影正算功能的实现}
有了类Spheroid的设计,我们先来完成高斯投影正算功能。
为了避免在计算中频繁的进行度分秒与弧度之间的转换问题,
我们设定除了特别声明之外,在类Spheroid中所用的角度均为弧度。

我们将高斯投影正算的函数名称定义为 BLtoXY,其设计如下:

 \begin{lstlisting}[language=C]
        /// <summary>
        /// 高斯投影正算
        /// </summary>
        /// <param name="B">纬度,单位:弧度</param>
        /// <param name="L">经度,单位:弧度</param>
        /// <param name="L0">中央子午线经度,单位:弧度</param>
        /// <param name="x">高斯平面x坐标</param>
        /// <param name="y">高斯平面y坐标</param>
        public void BLtoXY(double B, double L, double L0,
            out double x, out double y)
        {
            ......................................
        }
 \end{lstlisting}

由于函数的返回值为两个值x与y,无法以函数的返回值return的形式返回计算结果,
所以我们用函数参数 out 的形式将计算结果返回。

 由前面的高斯投影正算公式分析可知,高斯投影正算的计算较为简单,没有复杂的逻辑,
 先计算经差,然后计算子午线弧长后就可以直接写计算坐标x,y的算法了。
 但公式较为复杂,极容易写错,公式中具有大量的平方、四次方等变量。因此在编程时
 应将这些变量命名为与其相似的形式并提前计算。相关代码如下:
 \begin{lstlisting}[language=C]
        /// <summary>
        /// 计算子午线弧长
        /// </summary>
        /// <param name="B">纬度(单位:弧度)</param>
        /// <returns>子午线弧长</returns>
        private double funX(double B)
        {
            return a0 * B - a2 / 2.0 * Math.Sin(2 * B)
                + a4 / 4.0 * Math.Sin(4 * B)
                - a6 / 6.0 * Math.Sin(6 * B)
                + a8 / 8.0 * Math.Sin(8 * B);
        }

        private double funN(double sinB)
        {
            return a / Math.Sqrt(1 - e2 * sinB * sinB);
        }

        /// <summary>
        /// 高斯投影正算
        /// </summary>
        /// <param name="B">纬度,单位:弧度</param>
        /// <param name="L">经度,单位:弧度</param>
        /// <param name="L0">中央子午线经度,单位:弧度</param>
        /// <param name="x">高斯平面x坐标</param>
        /// <param name="y">高斯平面y坐标</param>
        public void BLtoXY(double B, double L, double L0,
            out double x, out double y)
        {
            double l = L - L0; //计算经差

            double sinB = Math.Sin(B);
            double cosB = Math.Cos(B);
            double cosB2 = cosB * cosB;
            double cosB4 = cosB2 * cosB2;
            double t = Math.Tan(B);
            double t2 = t * t;
            double t4 = t2 * t2;
            double g2 = eT2 * cosB * cosB;
            double g4 = g2 * g2;
            double l2 = l * l;
            double l4 = l2 * l2;

            double X = funX(B); //计算子午线弧长
            double N = funN(sinB);

           x = X + 0.5 * N * sinB * cosB * l2 * (1
                + cosB2 / 12.0 * (5 - t2 + 9 * g2 + 4 * g4) * l2
                + cosB4 / 360.0 * (61 - 58 * t2 + t4) * l4);


          y = N * cosB * l * (1
                + cosB2 * (1 - t2 + g2) * l2 / 6.0
                + cosB4 * (5 - 18 * t2 + t4 + 14 * g2 - 58 * g2 * t2)
                        * l4 / 120.0);
        }
 \end{lstlisting}

利用 BLtoXY 函数就可以进行高斯投影正算了,其算法流程为:
 \begin{lstlisting}[language=C]
    //创建克拉索夫斯基参考椭球
    Spheroid  spheroid = Spheroid.CreateBeiJing1954();

    //传入点的纬度B、经度与中央子午线经度,单位为弧度
    double B = ZXY.SMath.DMS2RAD(21.58470845);
    double L = ZXY.SMath.DMS2RAD(113.25314880);
    double L0 = ZXY.SMath.DMS2RAD(111);
    double x, y;

    spheroid.BLtoXY(B, L, L0, out x, out y);
\end{lstlisting}

点的纬度、经度为:$ B = 21 \degree 58'47.0845'',  L= 113\degree 25'31.4880''$,
中央子午线经度为:$L0 = 111\degree$,
计算出的坐标为:$x=2433586.692, y=250547.403$, 坐标y为点的自然坐标,
未加常数500km与带号。

\subsection{高斯投影反算功能的实现}
由高斯投影反算公式分析,在反算时需要首先计算底点纬度。底点纬度需要由
子午线弧长公式进行反算,由该公式可以看出,已知X=x时,这个函数在计算B值时它
并不是一个线型函数,不能直接计算。解决这类问题的计算方法就是迭代计算,我们将公式
进行变换,如下所示:

$$B= (X + \frac{a_2}{2}\sin 2B - \frac{a_4}{4}\sin 4B
+ \frac{a_6}{6} \sin 6B  - \frac{a_8}{8}\sin 8B)/a_0$$

在该公式中,两边都有B,我们将公式右边的B赋初始值$B_0=X/a_0$代入可以
计算出新的$B_i$值,循环进行计算,由于该迭代收敛,两值之差在一定范围内时
我们认为其值即为我们的解。

因此底点纬度计算函数设计为:
 \begin{lstlisting}[language=C]
        /// <summary>
        /// 计算底点纬度
        /// </summary>
        /// <param name="x">高斯平面x坐标</param>
        /// <returns>底点纬度</returns>
        private double funBf(double x)
        {
            double sinB, sin2B, sin4B, sin6B, sin8B;
            double Bf0 = x / a0, Bf = 0; //子午线弧长的初值
            int i = 0;
            while (i < 10000)//设定最大迭代次数
            {
                sinB = Math.Sin(Bf0);
                sin2B = Math.Sin(2 * Bf0);
                sin4B = Math.Sin(4 * Bf0);
                sin6B = Math.Sin(6 * Bf0);
                sin8B = Math.Sin(8 * Bf0);
                Bf = (x
                + a2 * sin2B / 2
                - a4 * sin4B / 4
                + a6 * sin6B / 6
                - a8 * sin8B / 8) / a0;

                if (Math.Abs(Bf - Bf0) < 1e-10) //计算精度
                    return Bf;
                else
                {
                    Bf0 = Bf;
                    i++;
                }
            }
            return -1e12;
        }
\end{lstlisting}

 在底点纬度计算出以后,高斯投影反算计算就没有难度了。
 我们将函数名命名为 XYtoBL,其函数设计为:
 \begin{lstlisting}[language=C]
        public void XY2BL(double x, double y, double L0,
            out double B, out double L)
        {
           double Bf = funBf(x);

            double cosBf = Math.Cos(Bf);
            double gf2 = eT2 * cosBf * cosBf;
            double gf4 = gf2 * gf2;
            double tf = Math.Tan(Bf);
            double tf2 = tf * tf;
            double tf4 = tf2 * tf2;
            double sinB = Math.Sin(Bf);
            double Nf = funN(sinB);
            double Mf = funM(sinB);
            double Nf2 = Nf * Nf;
            double Nf4 = Nf2 * Nf2;
            double y2 = y * y;
            double y4 = y2 * y2;

            B = Bf + tf * y2 / Mf / Nf * 0.5 * (
               -1.0
              + y2 * (5.0 + 3.0 * tf2 + gf2 - 9.0 * gf2 * tf2) / 12.0 / Nf2
              - y4 * (61.0 + 90.0 * tf2 + 45.0 * tf4) / 360.0 / Nf4);

           double l = y / Nf / cosBf * (
                1.0
              - y2 / 6.0 / Nf2 * (1.0 + 2.0 * tf2 + gf2)
             + y4 / 120.0 / Nf4
             * (5.0 + 28.0 * tf2 + 24.0 * tf4 + 6.0 * gf2 + 8.0 * gf2 * tf2));

            L = L0 + l;
        }
\end{lstlisting}

该函数中还有Nf与Mf需要计算,Nf的计算同前面的funN函数,Mf的计算函数为
 \begin{lstlisting}[language=C]
        private double funM(double sinB)
        {
            return a * (1 - e2) * Math.Pow(1 - e2 * sinB * sinB, -1.5);
        }
\end{lstlisting}

 有了高斯投影反算函数 XYtoBL, 就可以比较容易的写出其反算示例了,如以下代码所示:

\begin{lstlisting}[language=C]
     //创建克拉索夫斯基参考椭球
    Spheroid  spheroid = Spheroid.CreateBeiJing1954();

    //传入点的X、Y坐标与中央子午线经度(单位为弧度)
    double x=2433586.692, y=250547.403;
    double L0 = ZXY.SMath.DMS2RAD(111);

    double B, L;
    spheroid.XYtoBL(x, y, L0, out B, out L);
    //B= ZXY.SMath.RAD2DMS(B);
    //L= ZXY.SMath.RAD2DMS(L);
\end{lstlisting}

传入的y坐标应为真实坐标值,应不包括500km与带号等。
如果计算的经纬度需向界面展示,还应像上述代码后两行所示将其值转换为
度分秒形式。

计算出的点的纬度与经度为:
$B=21 \degree 58'47.0845'', L=113\degree 25'31.4880''$

计算点的子午线收敛角与计算某点处的长度变形值均比较简单,可以在正算或反算时将其同时
计算出,大家可以自己练习完成,在此就不再讲述了。

\subsection{换带计算}
换带计算其实质就是变换坐标系的中央子午线位置。因此首先应根据点的坐标
反算出点的经纬度,然后再根据新的坐标系中央子午线位置计算出点在
新的坐标系中的高斯平面坐标。

也就是先反算,再正算,注意此处的反算与正算的中央子午线经度值是不一样的。

该函数我们命名为 XYtoXY,其设计为如下代码:

\begin{lstlisting}[language=C]
        /// <summary>
        /// 高斯投影换带
        /// </summary>
        /// <param name="ox">点在源坐标系的x坐标</param>
        /// <param name="oy">点在源坐标系的y坐标</param>
        /// <param name="oL0">源坐标系的中央子午线经度,单位:弧度</param>
        /// <param name="nL0">目标坐标系的中央子午线经度,单位:弧度</param>
        /// <param name="nx">点在目标坐标系的x坐标</param>
        /// <param name="ny">点在目标坐标系的y坐标</param>
        public void XYtoXY(double ox, double oy,
            double oL0, double nL0,
           out double nx, out double ny)
        {
            double B, L;
            XYtoBL(ox, oy, oL0, out B, out L); //高斯投影反算
            BLtoXY(B, L, nL0, out nx, out ny); //高斯投影正算
        }
\end{lstlisting}

高斯投影换带的外部调用可以按如下方式写:

\begin{lstlisting}[language=C]
    double oldX = 3275110.535, oldY = 235437.233;

    double oldL0 = zx::CSurMath::DmsToRad(117);
    double newL0 = zx::CSurMath::DmsToRad(120);

    double newX, newY;
    Spheroid  spheroid = Spheroid.CreateBeiJing1954();
    spheroid.XYtoXY(oldX, oldY, oldL0, newL0, out newX, out newY);
\end{lstlisting}

计算出的点在新坐标系下的坐标为(3272782.315, -55299.545)。

至此,我们已经完成了高斯投影的全部计算功能了。需要注意的是以上的函数调用中的角度均
使用了弧度的形式,在外部调用时可以利用前面所讲的度分秒化弧度和弧度化度分秒函数先行转换。

\section{图形界面程序编写}
以上我们已经将主要的算法编写完毕,下面我们将利用 C\# 的 WPF技术编写图形界面,让我们的程序
变得更加实用。

\section{多点计算和文件读写}
以上我们的算法和测试验证程序均是针对一个点而言的,如果我们在一个文本形
式的数据文件里存放有多个点,怎么计算呢?

1、我们设计正算的数据文件格式为:
\begin{verbatim}
成果文件名

转换点个数 ~ 中央子午线经度

点名 ~ 大地纬度 ~ 大地经度
\end{verbatim}

示例数据文件为:

\begin{verbatim}
BLXY.txt
3  111
p1  21.58470845  113.25314880
p2  31.04416832  111.47248974
p3  30.45254425  111.17583596
\end{verbatim}

则设计函数为:
\begin{verbatim}
struct PntInfo
{
    char name[10];
    double B, L;
    double x, y;
};
void Zs()
{
    double L0;//所在带的中央子午线经度
    int n;//转换点的个数
    char cg[256];//成果文件名
    FILE *in;
    //读取文本文件的数据
    in = fopen("BLtoXY.txt", "r"); //打开已知文件
    fscanf(in,"%s",cg); //首先读入成果文件名称
    fscanf(in,"%d %lf", &n, &L0); //转换点的个数  中央子午线经度
    PntInfo * pnts = new PntInfo[n];//动态数组
    for(int i = 0; i < n; i++)//循环读入点名B、L
        fscanf(in, "%s %lf %lf", pnts[i].name, &pnts[i].B, &pnts[i].L);
    fclose(in);
    //以下进行正算计算
    CEarth * pEarth = CEarth::CreateEarth54();
    for(i = 0; i < n; i++)
        pEarth->GsZs(zx::CSurMath::DmsToRad(pnts[i].L),
         zx::CSurMath::DmsToRad(pnts[i].B),
             zx::CSurMath::DmsToRad(L0),
         pnts[i].x, pnts[i].y);
    //将计算后的数据写到成果文件中
    FILE * out;
    out = fopen(cg, "w");
    fprintf(out, "大地坐标(B,L)=====>国家坐标(X,Y)\n ");
    fprintf(out,"中央子午线经度: L0 = %lf\n", L0);
    fprintf(out,"序号 点名 B      L ===> X坐标(m)  Y坐标(m)\n");
    for(i = 0; i < n; i++)
    {
    fprintf(out," %3d %s %lf %lf %11.3f %11.3f\n",
        i+1, pnts[i].name, pnts[i].B, pnts[i].L,
            pnts[i].x, pnts[i].y );
    }
    fclose(out);
    delete[] pnts;//释放申请的内存
}
\end{verbatim}

同样也可实现多点的数据文件反算和换带计算

反算的数据文件格式设为:XYBL.txt
\begin{verbatim}
3   111
p1     2433586.692   250547.403
p2     3439978.970    75412.872
p3     3404139.839    28680.571
\end{verbatim}

换带计算的数据文件格式设为:
\begin{verbatim}
XYBL.txt
3   111 112
p1     2433586.692   250547.403
p2     3439978.970    75412.872
p3     3404139.839    28680.571
\end{verbatim}
相应的实现请参照正算实现。

\section{小结}
我们从一个逻辑结构不太明显的程序开始,将其改为一个结构较好的面向过程的
程序。并从一个算法开始逐步实现了整个程序。最后用面向对象的方法将其改写
并实现了全部功能。最后的程序从结构上看明显要比第一个程序要好,且更易维
护和扩充。当然高斯换带计算较为简单,我们的程序实现同样也较为简单。从这
个例子,我们知道了怎样设计结构良好的程序。
