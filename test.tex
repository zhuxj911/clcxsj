\documentclass[UTF8]{ctexart}
%% \documentclass{article}

\usepackage{graphicx} % 插图片

\usepackage{amsmath}  % 数学公式拓展
\usepackage{amssymb}

\usepackage{float}    % 浮动体控制

%% \usepackage{listings} % 插入代码,不过据说 minted 的效果更好一些
%% \usepackage{minted}

\usepackage{fontspec}

\setmainfont{Consolas}
%% \setCJKmainfont{微软雅黑}
\setCJKmainfont[BoldFont=黑体, ItalicFont=楷体]{宋体}
\setCJKfamilyfont{hei}{黑体}
\setCJKmonofont {仿宋}
\newcommand{\hei}{\CJKfamily{hei}}

\setCJKfamilyfont{hwxk}{STXingkai}             %使用STXingkai华文行楷字体
\newcommand{\huawenxingkai}{\CJKfamily{hwxk}}

\setCJKfamilyfont{hwcy}{STCaiyun}             %使用STCaiyun华文彩云字体
\newcommand{\huawencaiyun}{\CJKfamily{hwcy}}

\setCJKfamilyfont{hwhp}{STHupo}             %使用STHupo华文琥珀字体
\newcommand{\huawenhupo}{\CJKfamily{hwhp}}

\setCJKmainfont{宋体}


\usepackage{xcolor}
\usepackage{etoolbox}
\usepackage[colorlinks]{hyperref}
\usepackage{lipsum}
\usepackage[paperwidth=21cm,paperheight=29cm,top=2cm,bottom=2cm,left=3cm,right=3cm]{geometry}
\usepackage{xltxtra,mflogo,texnames}


\usepackage{minted}
\usepackage[user]{zref}
\usepackage{tcolorbox}

\newtcbox{\codelinenobox}[1][red]{on line,colback=#1!10!white,colframe=#1!50!black,
arc=4.5pt, %圆弧的半径
before upper={\rule[0pt]{0pt}{6pt}},
%调整字符在宽中的位置,[]中负的越多,字符越向上偏离底线
boxrule=0.5pt, %边框线宽
boxsep=0pt,%字符与边框的间距
left=2.5pt,right=2.5pt,top=1.5pt,bottom=1.5pt
}

\newcounter{codelineno}
\setcounter{codelineno}{0}
\renewcommand*{\thecodelineno}{\codelinenobox[blue]{\arabic{codelineno}}}

\makeatletter

\zref@newprop{lineno}{\thecodelineno}

\zref@addprops{main}{lineno}

\newcommand*{\codelineno}[2][]{%
\edef\ref@temp{#1}%
\ifx\ref@temp\@empty\relax%
\stepcounter{codelineno}%
\else%
\setcounter{codelineno}{#1}%
\fi%
\makebox[0pt][c]{\hss\thecodelineno\hspace{10pt}}%
\zlabel{#2}%
}


\makeatother

\begin{document}

\renewcommand{\theFancyVerbLine}{\sffamily
\textcolor{blue!50!black}{\small\oldstylenums{\arabic{FancyVerbLine}}}}

这是黑体加粗\textbf{加粗黑体}

\hei 这段文字是黑体

\textbf  加粗
\textit  意大利斜体
\textsl  slanted斜体
\textsf  无衬线
\texttt  等宽

\noindent 我是全局字体,我使用的是宋体\\
{\kaishu 我是ctex已定义好的字体,我使用的楷体}\\
{\heiti 我是ctex已定义好的字体,我使用的黑体}\\
{\fangsong 我是ctex已定义好的字体,我使用的仿宋}\\
{\lishu 我是ctex已定义好的字体,我使用的隶书}\\
{\youyuan 我是ctex已定义好的字体,我使用的幼圆}\\
{\huawenxingkai 我是自定义的字体,我使用的华文行楷}\\
{\huawencaiyun 我是自定义的字体,我使用的华文彩云}\\
{\huawenhupo 我是自定义的字体,我使用的华文琥珀}\\

Hello, \LaTeX

\begin{math}
 a^2 + b^2 = c^2
\end{math}

你好,\LaTeX

测试Vim + VimTex + MuPDF

\begin{minted}%
[encoding=utf8,
linenos,
frame=single,
rulecolor=purple!50!black,
texcl=true,
highlightcolor=green!40,
highlightlines={7,9},
]{python3}
#
#authoryear style
#
def formatlabelauthor(bibentry):

    if 'author' in bibentry:
        namelist=bibentry['author'] #\label{line:namelist:author}
    elif 'editor' in bibentry:
        namelist=bibentry['editor'] #\label{line:namelist:editor}
    elif 'translator' in bibentry:
        namelist=bibentry['translator']
    else:
        namelist='Anon'

    return [namelist,namelist]
\end{minted}

作者姓名赋值见第\ref{line:namelist:author}行,编者姓名赋值见\ref{line:namelist:editor}行。


\begin{minted}[escapeinside=||,
frame=single,
rulecolor=purple!50!black,
highlightcolor=yellow!50,
highlightlines={7,9},
]{python3}
#
#authoryear style
#
def formatlabelauthor(bibentry):

    if 'author' in bibentry:
|\codelineno{line:tex:d}|        namelist=bibentry['author'] 
    elif 'editor' in bibentry:
|\codelineno{line:tex:e}|         namelist=bibentry['editor'] 
    elif 'translator' in bibentry:
        namelist=bibentry['translator']
    else:
        namelist='Anon'

    return [namelist,namelist]
\end{minted}

作者姓名赋值见第\zref[lineno]{line:tex:d}行,编者姓名赋值见\zref[lineno]{line:tex:e}行。

在代码第7和9行,利用逃逸代码增加了新的行号标签,然后再正文中引用。标签和引用利用zref宏包实现。
行号标签的用tcolorbox画了一个圈。

定义的 \verb|\codelineno|命令,根据输入信息完成了行号的更新,行号标签的输出,以及标签生成。
注意python代码中的 \verb|\codelineno{line:tex:d}|。

$$
\mathbb{ABCDEFGHIJKLMNOPQRSTUVWXYZ}
\mathbf{012...abc...ABC}
$$

\[
	A=\bigcap_{i \in I } 	
.\]

朱学军
西安科技大学测绘科学与技术学院
tel:13892885268
email:xazhuxj@qq.com

2020-09-18

\end{document}
